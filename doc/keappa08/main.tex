%%% -*- mode: LaTeX; TeX-master: t; -*-
%%% 
%%% Please see the README file for usage information.

%%% PLEASE DO NOT EDIT THE FOLLOWING BIT
\let\master\relax
\documentclass[a4paper]{easychair}
%%% PLEASE DO NOT EDIT THE PRECEDING BIT
 
%%% remove [draft] to suppress editorial notes, etc.
%%%    DO THIS BEFORE SUBMISSION
%%
\usepackage{submission}  % contains all macros
% \usepackage[draft]{submission}  % contains all macros
%%

\begin{document}

%%% -*- mode: LaTeX; TeX-master: "main.tex"; -*-

\ifx\master\undefined
\documentclass[a4paper]{easychair}
\usepackage{submission}
\begin{document}
\fi
%%%% PLEASE DO NOT EDIT ABOVE THIS LINE

\title{A \tlaplus Proof System}

\titlerunning{A \tlaplus Proof System}

% \volumeinfo
% 	{P. Rudnicki, G. Sutcliffe} % editors
% 	{2}                         % number of editors
% 	{KEAPPA 2008}               % event
% 	{1}                         % volume
% 	{1}                         % issue
% 	{1}                         % starting page number


%  Alphabetically by surname
\author{
  Kaustuv Chaudhuri \\
  INRIA \\
  \and
  Damien Doligez \\
  INRIA \\
  \and
  Leslie Lamport \\
  Microsoft Research \\
  \and
  Stephan Merz \\
  INRIA \& Loria
}

\authorrunning{Chaudhuri, Doligez, Lamport, and Merz}

\maketitle

%%%% PLEASE DO NOT EDIT BELOW THIS LINE
\ifx\master\undefined
{\let\master\relax %%% -*- mode: LaTeX; TeX-master: "main.tex"; -*-

\ifx\master\undefined
\documentclass[a4paper]{easychair}
\usepackage{submission}
\begin{document}
{\let\master\relax %%% -*- mode: LaTeX; TeX-master: "main.tex"; -*-

\ifx\master\undefined
\documentclass[a4paper]{easychair}
\usepackage{submission}
\begin{document}
\fi
%%%% PLEASE DO NOT EDIT ABOVE THIS LINE

\title{A \tlaplus Proof System}

\titlerunning{A \tlaplus Proof System}

% \volumeinfo
% 	{P. Rudnicki, G. Sutcliffe} % editors
% 	{2}                         % number of editors
% 	{KEAPPA 2008}               % event
% 	{1}                         % volume
% 	{1}                         % issue
% 	{1}                         % starting page number


%  Alphabetically by surname
\author{
  Kaustuv Chaudhuri \\
  INRIA \\
  \and
  Damien Doligez \\
  INRIA \\
  \and
  Leslie Lamport \\
  Microsoft Research \\
  \and
  Stephan Merz \\
  INRIA \& Loria
}

\authorrunning{Chaudhuri, Doligez, Lamport, and Merz}

\maketitle

%%%% PLEASE DO NOT EDIT BELOW THIS LINE
\ifx\master\undefined
{\let\master\relax \input{rearmatter}}
\end{document}
\fi

% LocalWords:  tex Rudnicki Sutcliffe KEAPPA Kaustuv Chaudhuri INRIA Doligez
% LocalWords:  Merz Loria
}
\fi
%%%% PLEASE DO NOT EDIT ABOVE THIS LINE

\bibliographystyle{plain}
\bibliography{submission}

%%%% PLEASE DO NOT EDIT BELOW THIS LINE
\ifx\master\undefined
\end{document}
\fi

% LocalWords:  tex Paxos
}
\end{document}
\fi

% LocalWords:  tex Rudnicki Sutcliffe KEAPPA Kaustuv Chaudhuri INRIA Doligez
% LocalWords:  Merz Loria
             % title, list of authors

% \llnote{We have been inconsistent about what we call the prover.  
%
% \ednote{SM}{Indeed, we should be consistent. However, remember that
%   the proofs found by the back-end(s) are in turn supposed to be
%   checked by Isabelle, so ``checker'' may be a bit confusing.}
%
% \ednote{DD}{This has been resolved: we have ``back-end provers''
% (Zenon and Isabelle tactics, later also haRVey) that ``verify'' proof
% obligations, and a ``certifier'' (Isabelle/TLA+) that ``certifies''
% their output} 
% I have found some places where the proper notation is still not
% being used.  I corrected all that I found, but I probably missed
% some.  Please check.\\[.3em]
% I also discovered that we have been writing ``\PM'' when we mean the
% system consisting of the \PM\ together with its back-end provers.
% (For example, we talk about the \PM\ checking a proof.)  I have used
% the term ``proof system'' to mean the \PM\ plus the back-end provers,
% and have introduced it where necessary.  But I've probably missed
% some places.
% }

\ifdraft
\begin{center}
\large\today
\end{center}
\fi

%%% -*- mode: LaTeX; TeX-master: "main.tex"; -*-

\ifx\master\undefined
\documentclass[a4paper]{easychair}
\usepackage{submission}
\begin{document}
{\let\master\relax%%% -*- mode: LaTeX; TeX-master: "main.tex"; -*-

\ifx\master\undefined
\documentclass[a4paper]{easychair}
\usepackage{submission}
\begin{document}
\fi
%%%% PLEASE DO NOT EDIT ABOVE THIS LINE

\title{A \tlaplus Proof System}

\titlerunning{A \tlaplus Proof System}

% \volumeinfo
% 	{P. Rudnicki, G. Sutcliffe} % editors
% 	{2}                         % number of editors
% 	{KEAPPA 2008}               % event
% 	{1}                         % volume
% 	{1}                         % issue
% 	{1}                         % starting page number


%  Alphabetically by surname
\author{
  Kaustuv Chaudhuri \\
  INRIA \\
  \and
  Damien Doligez \\
  INRIA \\
  \and
  Leslie Lamport \\
  Microsoft Research \\
  \and
  Stephan Merz \\
  INRIA \& Loria
}

\authorrunning{Chaudhuri, Doligez, Lamport, and Merz}

\maketitle

%%%% PLEASE DO NOT EDIT BELOW THIS LINE
\ifx\master\undefined
{\let\master\relax %%% -*- mode: LaTeX; TeX-master: "main.tex"; -*-

\ifx\master\undefined
\documentclass[a4paper]{easychair}
\usepackage{submission}
\begin{document}
{\let\master\relax \input{frontmatter}}
\fi
%%%% PLEASE DO NOT EDIT ABOVE THIS LINE

\bibliographystyle{plain}
\bibliography{submission}

%%%% PLEASE DO NOT EDIT BELOW THIS LINE
\ifx\master\undefined
\end{document}
\fi

% LocalWords:  tex Paxos
}
\end{document}
\fi

% LocalWords:  tex Rudnicki Sutcliffe KEAPPA Kaustuv Chaudhuri INRIA Doligez
% LocalWords:  Merz Loria
}
\fi
%%%% PLEASE DO NOT EDIT ABOVE THIS LINE

\begin{abstract}
  We describe an extension to the \tlaplus{} specification language
  with constructs for writing proofs and a proof environment, called
  the Proof Manager (PM), that checks those proofs.  The language and
  the \PM\ support the incremental development and checking of
  hierarchically structured proofs.  The \PM\ translates a proof into
  a set of independent proof obligations and calls upon a collection
  of back-end provers to verify them.  Different provers can be used
  to verify different obligations.  The currently supported back-ends
  are the tableau prover Zenon and Isabelle/\tlaplus{}, an
  axiomatisation of \tlaplus\ in Isabelle/Pure.  Isabelle/\tlaplus{}
  can also certify the correctness of a complete proof.
\end{abstract}


%%%% PLEASE DO NOT EDIT BELOW THIS LINE
\ifx\master\undefined
\end{document}
\fi
                % 

% \llnote{We should submit the paper with a table of contents.  A table of
% contents helps the reader.  It is traditionally omitted from 
% articles in journals and proceedings because it was hard to make
% them when the journal or proceedings was typeset separately from
% the typesetting of the individual articles.  It is stupid to omit
% a table of contents in an article that is not part of a journal
% or proceedings---even if it's a preprint of an article that will appear
% in one.  The same goes for a date.\\[.3em]
% In our case, creating this separate, unnumbered page also effectively
% excludes the abstract from the 10-page limit.
% }

\ednote{KC}{I moved the TOC to the end of the paper (in non-draft mode
  only). It is the French style, and it moreover allows all
  supplemental material to be numbered differently. I am not a huge
  fan of arguing with the PC about page limits. The paper without
  ednotes as of this commit fits snugly in 10 pages.
\llnote{And what happened to the statement saying that the appendix is
for the reviewers use, and we are not committed to including it in the
final version?}
}

\ifdraft
\thispagestyle{empty}
\tableofcontents
\setcounter{page}{0}
\fi

%%% -*- mode: LaTeX; TeX-master: "main.tex"; -*-

\ifx\master\undefined
\documentclass[a4paper]{easychair}
\usepackage{submission}
\begin{document}
{\let\master\relax%%% -*- mode: LaTeX; TeX-master: "main.tex"; -*-

\ifx\master\undefined
\documentclass[a4paper]{easychair}
\usepackage{submission}
\begin{document}
\fi
%%%% PLEASE DO NOT EDIT ABOVE THIS LINE

\title{A \tlaplus Proof System}

\titlerunning{A \tlaplus Proof System}

% \volumeinfo
% 	{P. Rudnicki, G. Sutcliffe} % editors
% 	{2}                         % number of editors
% 	{KEAPPA 2008}               % event
% 	{1}                         % volume
% 	{1}                         % issue
% 	{1}                         % starting page number


%  Alphabetically by surname
\author{
  Kaustuv Chaudhuri \\
  INRIA \\
  \and
  Damien Doligez \\
  INRIA \\
  \and
  Leslie Lamport \\
  Microsoft Research \\
  \and
  Stephan Merz \\
  INRIA \& Loria
}

\authorrunning{Chaudhuri, Doligez, Lamport, and Merz}

\maketitle

%%%% PLEASE DO NOT EDIT BELOW THIS LINE
\ifx\master\undefined
{\let\master\relax %%% -*- mode: LaTeX; TeX-master: "main.tex"; -*-

\ifx\master\undefined
\documentclass[a4paper]{easychair}
\usepackage{submission}
\begin{document}
{\let\master\relax \input{frontmatter}}
\fi
%%%% PLEASE DO NOT EDIT ABOVE THIS LINE

\bibliographystyle{plain}
\bibliography{submission}

%%%% PLEASE DO NOT EDIT BELOW THIS LINE
\ifx\master\undefined
\end{document}
\fi

% LocalWords:  tex Paxos
}
\end{document}
\fi

% LocalWords:  tex Rudnicki Sutcliffe KEAPPA Kaustuv Chaudhuri INRIA Doligez
% LocalWords:  Merz Loria
}
\fi
%%%% PLEASE DO NOT EDIT ABOVE THIS LINE

\section{Introduction}
\label{sec:intro}

\tlaplus\ is a language for specifying the behaviour of concurrent and
distributed systems and asserting properties of those
systems~\cite{lamport03tla}.  However, it provides no way to write
proofs of those properties.  We have designed an extended version of
the language that allows writing proofs, and we have begun
implementing a system centered around a \textit{Proof Manager} (\PM)
that invokes existing automated and interactive proof systems to check
those proofs. For now, the new version of \tlaplus\ is called \tlatwo\
to distinguish it from the current one.  We describe here the \tlatwo
proof constructs and the current state of the proof system.

The primary goal of \tlatwo\ and the proof system is the mechanical
verification of systems specifications. The proof system must not only
support the modal and temporal aspects of TLA needed
to reason about system properties, but must also support
ordinary mathematical reasoning in the underlying logic. Proofs in
\tlatwo are natural deduction proofs written in a hierarchical style
that we have found to be good for ordinary
mathematics~\cite{lamport93amm} and crucial for managing the
complexity of correctness proofs of systems~\cite{gafni:disk-paxos}.

The \PM\ computes proof obligations that establish the correctness of
the proof and sends them to one or more back-end provers to be
verified.  Currently, the back-end provers are Isabelle/\tlaplus, a
faithful axiomatisation of \tlaplus\ in Isabelle/Pure, and
Zenon~\cite{bonichon07lpar}, a tableau prover for classical
first-order logic with equality.  The \PM\ first sends a proof
obligation to Zenon.  If Zenon succeeds, it produces an Isar script
that the \PM\ sends to Isabelle to certify.  Otherwise, the \PM
directs Isabelle to try to find a proof with its automated tactics.
In both cases, the proofs are certified by Isabelle/\tlaplus.
%%
%% SM: I changed the following sentence because its original
%% formulation seems to defeat certifiability, an issue that we
%% highlight.
%%
%This system
%architecture allows us in the future to add other back-end provers
%whose proofs need not be certified by Isabelle/\tlaplus.
The system architecture easily accomodates other back-end provers; if
these are proof-producing, certification by Isabelle/\tlaplus\ gives
us high high confidence in the overall correctness.

The \tlatwo\ proof constructs are described in
Section~\ref{sec:proof-language}.  Section~\ref{sec:obligations}
describes the proof obligations generated by the \PM, and
Section~\ref{sec:backend} describes how the \PM\ uses Zenon and
Isabelle to verify them.  The conclusion summarizes what we have done
and not yet done and briefly discusses related work.

%%%% PLEASE DO NOT EDIT BELOW THIS LINE
\ifx\master\undefined
{\let\master\relax%%% -*- mode: LaTeX; TeX-master: "main.tex"; -*-

\ifx\master\undefined
\documentclass[a4paper]{easychair}
\usepackage{submission}
\begin{document}
{\let\master\relax %%% -*- mode: LaTeX; TeX-master: "main.tex"; -*-

\ifx\master\undefined
\documentclass[a4paper]{easychair}
\usepackage{submission}
\begin{document}
\fi
%%%% PLEASE DO NOT EDIT ABOVE THIS LINE

\title{A \tlaplus Proof System}

\titlerunning{A \tlaplus Proof System}

% \volumeinfo
% 	{P. Rudnicki, G. Sutcliffe} % editors
% 	{2}                         % number of editors
% 	{KEAPPA 2008}               % event
% 	{1}                         % volume
% 	{1}                         % issue
% 	{1}                         % starting page number


%  Alphabetically by surname
\author{
  Kaustuv Chaudhuri \\
  INRIA \\
  \and
  Damien Doligez \\
  INRIA \\
  \and
  Leslie Lamport \\
  Microsoft Research \\
  \and
  Stephan Merz \\
  INRIA \& Loria
}

\authorrunning{Chaudhuri, Doligez, Lamport, and Merz}

\maketitle

%%%% PLEASE DO NOT EDIT BELOW THIS LINE
\ifx\master\undefined
{\let\master\relax \input{rearmatter}}
\end{document}
\fi

% LocalWords:  tex Rudnicki Sutcliffe KEAPPA Kaustuv Chaudhuri INRIA Doligez
% LocalWords:  Merz Loria
}
\fi
%%%% PLEASE DO NOT EDIT ABOVE THIS LINE

\bibliographystyle{plain}
\bibliography{submission}

%%%% PLEASE DO NOT EDIT BELOW THIS LINE
\ifx\master\undefined
\end{document}
\fi

% LocalWords:  tex Paxos
}
\end{document}
\fi
            % 

%%% -*- mode: LaTeX; TeX-master: "main.tex"; -*-

\ifx\master\undefined
\documentclass[a4paper]{easychair}
\usepackage{submission}
\begin{document}
{\let\master\relax%%% -*- mode: LaTeX; TeX-master: "main.tex"; -*-

\ifx\master\undefined
\documentclass[a4paper]{easychair}
\usepackage{submission}
\begin{document}
\fi
%%%% PLEASE DO NOT EDIT ABOVE THIS LINE

\title{A \tlaplus Proof System}

\titlerunning{A \tlaplus Proof System}

% \volumeinfo
% 	{P. Rudnicki, G. Sutcliffe} % editors
% 	{2}                         % number of editors
% 	{KEAPPA 2008}               % event
% 	{1}                         % volume
% 	{1}                         % issue
% 	{1}                         % starting page number


%  Alphabetically by surname
\author{
  Kaustuv Chaudhuri \\
  INRIA \\
  \and
  Damien Doligez \\
  INRIA \\
  \and
  Leslie Lamport \\
  Microsoft Research \\
  \and
  Stephan Merz \\
  INRIA \& Loria
}

\authorrunning{Chaudhuri, Doligez, Lamport, and Merz}

\maketitle

%%%% PLEASE DO NOT EDIT BELOW THIS LINE
\ifx\master\undefined
{\let\master\relax %%% -*- mode: LaTeX; TeX-master: "main.tex"; -*-

\ifx\master\undefined
\documentclass[a4paper]{easychair}
\usepackage{submission}
\begin{document}
{\let\master\relax \input{frontmatter}}
\fi
%%%% PLEASE DO NOT EDIT ABOVE THIS LINE

\bibliographystyle{plain}
\bibliography{submission}

%%%% PLEASE DO NOT EDIT BELOW THIS LINE
\ifx\master\undefined
\end{document}
\fi

% LocalWords:  tex Paxos
}
\end{document}
\fi

% LocalWords:  tex Rudnicki Sutcliffe KEAPPA Kaustuv Chaudhuri INRIA Doligez
% LocalWords:  Merz Loria
}
\fi
%%%% PLEASE DO NOT EDIT ABOVE THIS LINE

\section{\tlaplus and its Proof Language}
\label{sec:proof-language}

\subsection{TLA}
\label{sec:proof-language.tla} 

The \tlaplus\ language is based on the Temporal Logic of Actions
(TLA)~\cite{lamport:newtla}, a linear-time temporal logic. The rigid
variables of TLA are called \emph{constants} and the flexible
variables are called simply \emph{variables}.  TLA assumes an
underlying ordinary (non-modal) logic for constructing expressions.
Operators of that logic are called \emph{constant} operators.  A
\emph{state function} is an expression built from constant operators
and TLA constants and variables.  The elementary (non-temporal)
formulas of TLA are \textit{actions}, which are formulas built with
constant operators, constants, variables, and expressions of the form
$f'$, where $f$ is a state function.  (TLA also has an \ENABLED
operator that is used in reasoning about liveness, but we ignore it
for brevity.)  An action is interpreted as a predicate on pairs of
states that describes a set of possible state transitions, where state
functions refer to the starting state and primed state functions refer
to the ending state.  Because priming distributes over constant
operators and because $c'$ is equal to $c$ for any constant $c$, an
action can be reduced to a formula built from constant operators,
constants, variables, and primed variables.

TLA is practical for describing systems because all the complexity of
a specification is in the action formulas.  Temporal operators are
essentially used only to assert liveness properties, including
fairness of system actions.  Most of the work in a TLA proof is in
proving action formulas; temporal reasoning occurs only in proving
liveness properties and is limited to propositional temporal logic and
to applying a handful of proof rules whose main premises are action
formulas.  Because temporal reasoning is such a small part of TLA
proofs, we have deferred its implementation.  The \PM\ now handles
only action formulas.  We have enough experience mechanizing TLA's
temporal reasoning~\cite{engberg:mechanical} to be fairly confident
that it will not be hard to extend the \PM to support it.

A formula built from constant operators, constants, variables, and
primed variables is valid iff it is a valid formula of the underlying
logic when constants, variables, and primed variables are treated as
distinct variables of the logic---that is, if $v$ and $v'$ are
considered to be two distinct variables of the underlying logic, for
any TLA variable $v$.  Since any action formula is reducible to such a
formula, action reasoning is immediately reducible to reasoning in the
underlying logic.  We therefore ignore variables and priming here and
consider only constant formulas.


\subsection{\tlaplus}

The \tlaplus\ language adds the following to the TLA logic:
\begin{icom}
\item An underlying logic that is essentially ZFC set theory plus
  classical untyped first-order logic with Hilbert's
  $\varepsilon$~\cite{leisenring:mathematical-logic}.
% LL: I deleted
% 
%    which is written \CHOOSE\ in \tlaplus.  
% 
% to save a line and avoid unnecessarily introducing a name.
% 
The major difference between this underlying
  logic and traditional ZFC is that functions are defined axiomatically 
  rather than being represented as sets of ordered pairs.

\item A mechanism for defining operators, where a user-defined
operator is essentially a macro that is expanded syntactically.
(\tlaplus\ permits recursive function definitions, but they are
translated to ordinary definitions using Hilbert's $\varepsilon$.)

%% LL: changed ``A module language'' to ``Modules''
\item Modules, where one module can import definitions
  and theorems from other modules.  A module is parameterized by its
  declared variables and constants, and it may be instantiated in another
  module by substituting expressions for its parameters. The
  combination of substitution and the \ENABLED\ operator introduces
  some complications, but space limitations prevent us from discussing
  them, so we largely ignore modules in this paper.
\end{icom}
\tlaplus\ has been extensively documented~\cite{lamport03tla}.  Since
we are concerned only with reasoning about its underlying logic, which
is a very familiar one, we do not bother to describe \tlaplus\ in any
detail.  All of its nonstandard notation that appears in our examples is
explained.

\subsection{The Proof Language}
\label{sec:proof-language.lang}

The major new feature of \tlatwo\ is its proof language.  (For reasons
having nothing to do with proofs, \tlatwo\ also introduces recursive
operator definitions, which we ignore here for brevity.)  We describe
the basic proof language, omitting a few constructs
%%
% LL: I don't think we omitted any ``uninteresting'' constructs,
%     so I deleted:
% 
% 
%    for simplicity a few constructs that are either uninteresting or
%    
% 
that concern aspects such as module instantiation that we are not
discussing.  \tlatwo\ also adds constructs for naming subexpressions
of a definition or theorem, which is important in practice for writing
proofs but is orthogonal to the concerns of this paper.

The goal of the language is to make proofs easy to read and write for
someone with no knowledge of how the proofs are being checked.  This
leads to a mostly declarative language, built around the uses and
proofs of assertions rather than around the application of
proof-search tactics.  It is therefore more akin to
Isabelle/Isar~\cite{isar} than to more operational interactive
languages such as Coq's Vernacular~\cite{coq}.
Nevertheless, the proof language does include a few operational
constructs that can eliminate the repetition of common idioms, albeit
with some loss of perspicuity.

% at the cost of some loss of perspicuity.


At any point in a \tlaplus\ proof, there is a current obligation that
is to be proved.  The obligation contains a \emph{context} of
\emph{known} facts, definitions, and declarations, and a
\emph{goal}.
%%
%% SM: slight rewording of the following
%%
% which is a proposition that the obligation claims to be entailed by
% the context.
The obligation claims that the goal is logically entailed by the context.
Some of the facts and definitions in the
context are marked as \emph{usable} for reasoning, while the remaining
facts and definitions are \textit{hidden}.
%
%% LL: We'd better say more about this below, so it's redundant
%% here:
% 
% and may not be used in the
% proof unless explicitly made usable with a \USE declaration (see
% below).

\ednote{SM}{I don't think we are quite consistent about what is an
  assertion. This presentation suggests that an assertion is not just
  an \ASSUME{} \PROOF{}, but also ``contains'' the context in which
  that \ASSUME{} \PROOF{} appears. I think that the term ``assertion''
  is used differently towards the end of this section, although the
  description here matches what we say in Section 3.}

Proofs are structured hierarchically.
%
%% LL: This is deferred until the first use of PROOF.
% and optionally begin with the token \PROOF. 
%
% At the lowest levels of the hierarchy are the
% \textit{leaf proofs}.  
The leaf (lowest-level) proof \OBVIOUS\ asserts that the
current goal follows easily from the usable facts and definitions.
The leaf proof
\begin{gather*}
  \BY\ e_{1},\ldots, e_{m} \ \DEFS\ o_{1},\ldots, o_{n}
\end{gather*}
asserts that the current goal follows easily from the usable facts and
definitions together with (i)~the facts $e_{i}$ that must themselves
follow easily from the context and (ii)~the known definitions of
$o_{j}$.  Whether a goal follows easily from definitions and facts
depends on who is trying to prove it.  The \PM\ generates proof
obligations for each leaf proof, so in practice ``follows easily''
means that a back-end prover can prove them.

A non-leaf proof is a sequence of \textit{steps}, each consisting of a
begin-step token and a proof construct.  For some constructs
(including a simple assertion of a proposition) the step takes a
subproof, which may be omitted.  The final step in the sequence simply
asserts the current goal, which is represented by the token \QED.
%
A begin-step token is either a \emph{level token} of the form \s{n} or
a \emph{label} of the form \s{n}"l", where $n$ is a level number that
is the same for all steps of this non-leaf proof, and "l" is an
arbitrary name.  The hierarchical structure is deduced from the level
numbers of the begin-step tokens, a higher level number beginning a
subproof.

%% LL: I changed 
%   A step that makes declarations or definitions or that changes the form
%   of the current goal does not require a proof,
%   but one that makes an assertion is followed by its proof 
%
% because (a) SUFFICES changes the goal and requires a proof and 
% (b) PICK makes a definition and requires a proof
% (c) WITNESS e \in S makes an assertion but does not take a proof.
%  Also assertions whose proof is omitted are not followed by their proof--
%  at least not with the reader's current understanding of proof.
%
Some steps make declarations or definitions or change the current
goal and do not require a proof.  Other steps make assertions that
become the current goals for their proofs.  An omitted proof (or one
consisting of the token \OMITTED) is considered to be a leaf proof
that instructs the assertion to be accepted as true.  Of course, the
proof is then incomplete and cannot be certified.  Following a step
that makes an assertion (and the step's proof), until the end of the
current proof (after the \QED\ step), the contexts contain that
assertion in their sets of known facts.  The assertion is marked
usable iff the begin-step token is a level token; otherwise it can be
referred to by its label in a \BY\ proof.

The hierarchical structure of proofs not only aids in reading finished
proof but is also quite useful in incrementally writing proofs.  The
steps of a non-leaf proof are first written with all proofs but that
of the \QED\ step omitted.  After checking the proof of the \QED
step, the proofs omitted for other steps in this or earlier levels
are written in any order.  When writing the proof, one may discover
facts that are needed in the proofs of multiple steps.
% 
% LL: I modified the following text because we say in conclusion
%     that adding lemmas is bad. 
%
Such a fact is then added to the proof as an early step, or added at a
higher level.  It can also be removed from the proof of the theorem
and proved separately as a lemma.  However, the hierarchical proof
language encourages facts relevant only for a particular proof to be
kept within the proof, making the proof's structure easier to see and
simplifying maintenance of the proof.  For correctness proofs of
systems, the first few levels of the hierarchy are generally
determined by the structure of the formula to be proved---for example,
the proof that a formula implies a conjunction usually consists of steps
asserting that it implies each conjunct.

As an example, we incrementally construct a hierarchical proof of
Cantor's theorem, which states that there is no surjective function
from a set to its powerset. It is written in \tlaplus as:
%
\begin{quote} \small
  \begin{tabbing}
    \THEOREM\ "\forall S : \forall f \in [S -> \SUBSET\ S] : 
        \exists A \in \SUBSET\ S : \forall x \in S : f[x] \neq A"
  \end{tabbing}
\end{quote}
%
%% SM: cosmetic change of the following that heeds Leslie's advice below
%% (I think it used to be more or less like that before, why did it
%% change?)
%%
%where "f[x]" is the application of the function "f" to "x",
where function application is written using square brackets,
"\SUBSET\ S" is the powerset of "S", and "[S -> T]" is the set of
functions from $S$ to $T$.
% \llnote{One should not write two mathematical operators like
% $x$ and $\SUBSET$ separated only by punctuation, since that is hard to
% read.  The sentence above needs to be rewritten.}

The statement of the theorem is the current goal for 
its top-level proof. A goal of the form $\forall v:e$ is
proved by introducing a generic constant and proving the formula
obtained by 
substituting it for the bound identifier. We express this as follows,
using the \ASSUME/\PROVE construct of \tlatwo:
% \llnote{\ASSUME/\PROVE should be introduced earlier, where we mention
% assertions---probably back where we introduce "||-".}

\begin{quote} \small
  \begin{tabbing}
    \THEOREM\ "\forall S : \forall f \in [S -> \SUBSET\ S] : 
                \exists A \in \SUBSET\ S : \forall x \in S : f[x] \neq A" \\
%%    \PROOF \\
    \LSP \= \s11.\ \= \ASSUME \= "\NEW\ S", \\
         \>        \>         \> "\NEW\ f \in [S -> \SUBSET\ S]"\\
         \>        \> \PROVE "\exists A \in \SUBSET\ S : \forall x \in S : f[x] \neq A" \\
         \> \s12.  \> \QED \BY \s11
  \end{tabbing}
\end{quote}
%
%
Step \s11 asserts that for any constants "S" and "f" with "f \in [S
-> \SUBSET\ S]", the proposition to the right of the \PROVE is true.
More precisely, the current context for the proof of \s11 (which we
have yet to write) contains the declarations of $S$ and $f$ and the
usable fact "f \in [S -> \SUBSET\ S]", and the \PROVE\ assertion is
its goal.  The \QED step states that the original goal (the theorem)
follows from the assertion in step~\s11.

We tell the \PM to check this (incomplete) proof, which it does by
having the back-end provers verify the proof obligation for the \QED
step.  The verification succeeds, and we now continue by writing the
proof of \s11.  (Had the verification failed because \s11 did not
imply the current goal, we would have caught the error before
attempting to prove \s11, which we expect to be harder to do.)

\llnote{Check minor edit below.}

We optimistically start with the proof \OBVIOUS, but it is too hard
for the back-end to prove, and the \PM reports a timeout.  Often this
means that a necessary fact or definition in the context is hidden and
we merely have to make it usable with a \USE step or a \BY proof.  In
this case we have no such hidden assumptions, so we must refine the
goal into simpler goals with a non-leaf proof.  We let this proof have
level 2 (we can use any level greater than 1).  Since the goal itself
is existentially quantified, we must supply a witness.  In this case,
the witness is the classic diagonal set, which we call~"T".
%
\begin{quote} \small
  \begin{tabbing}
    \THEOREM\ "\forall S : \forall f \in [S -> \SUBSET\ S] : \exists A \in \SUBSET\ S : \forall x \in S : f[x] \neq A" \kill
    \PROOF \kill
    \LSP \= \s11.\ \= \ASSUME \= "\NEW\ S", \\
         \>        \>         \> "\NEW\ f \in [S -> \SUBSET\ S]" \\
         \>        \> \PROVE "\exists A \in \SUBSET\ S : \forall x \in S : f[x] \neq A" \\
%         \>        \> \PROOF \\
         \>   \hspace{1em}     \= \s21.\ \= \DEFINE "T \DEF \{z \in S : z \notin f[z]\}" \\
         \>        \> \s22.  \> "\forall x \in S : f[x] \neq T" \\
         \>        \> \s23.  \> \QED \BY \s22
  \end{tabbing}
\end{quote}
Because definitions made within a proof are usable by default, the
definition of $T$ is usable in the proofs of \s22\ and \s23.  Once
again, the proof of the \QED\ step is automatically verified, so all
that remains is to prove \s22.  (The \DEFINE\ step requires no proof.)

% \llnote{I added to the paragraph above an explanation of why the
%   definition of $T$ is usable.  Stephan asks if it wouldn't it be more
%   consistent to give the definition a level token.  In a comment that
%   got deleted along the way, I proposed changing the language to make
%   labeled definitions unusable and allow the user to write $\BY\
%   \DEFS\ \s21$ as well as $\BY\ \DEFS\ T$.  However, we had already
%   decided that the user is most likely to want local definitions to be
%   \USE{}d by default, and many users might prefer to give step numbers
%   to all their steps.  So, we decided to defer a decision on this
%   point.  In the conclusion, I added this as the kind of fine tuning
%   we are likely to make to the language.}
% 
% \ednote{SM}{Fine with me. There must be a way to hide definitions
%   because they can get in the way. Your suggestion to make labeled
%   definitions unusable looks clean and consistent to me, but in its
%   absence we can at least write HIDE DEF.}
% 
% \ednote{KC}{In my opinion, the presence or absence of a label should
%   not be the deciding factor on whether the contents of the step, be
%   they definitions or facts, are usable or hidden. Instead, the label
%   should determine whether the contents of the step are namable by the
%   step token or not. Since definitions are not namable, it does not
%   matter whether we number "\DEFINE" steps or not. (Subexpressions of
%   the body of the definition are reachable via the defined operator
%   name.)}

%
% \llnote{This example suggests that we should modify the current language
% definition so that definitions in a labeled step are not automatically
% usable.  There's no reason why they need to be, and it would make it
% clear that the \QED\ step's proof doesn't depend on the definition.
% We could allow BY DEF \s21\ to include all the definitions made
% in the step.}

The system 
%% LL: changed ``will accept'' to ``accepts'' because it does so right now.
accepts \OBVIOUS\ as the proof of \s22 because the only
difficulty in the proof of \s11 is finding the
witness. However, suppose we want to add another level of proof for the benefit
of a human reader.  The universal quantification is proved as above,
by introducing a fresh constant:

\begin{quote} \small
  \begin{tabbing}
    \THEOREM\ "\forall S : \forall f \in [S -> \SUBSET\ S] : \exists A \in \SUBSET\ S : \forall x \in S : f[x] \neq A" \kill
    \PROOF \kill
    \LSP \= \s11.\ \= \ASSUME \= "\NEW\ S", \kill
         \>        \>         \> "\NEW\ f \in [S -> \SUBSET\ S]" \kill
         \>        \> \PROVE "\exists A \in \SUBSET\ S : \forall x \in S : f[x] \neq A" \kill
         \>        \> \PROOF \kill
         \>   \hspace{1em} \= \s21.\ \= \DEFINE "T == \{z \in S : z \notin f[z]\}" \kill
         \>        \> \s22.  \> "\forall x \in S : f[x] \neq T" \\
         \>        \> \hspace{1em} \= \s31.\ \= \ASSUME "\NEW\ x \in S" \PROVE "f[x] \neq T" \\
         \>        \>        \> \s32.\ \> \QED \BY \s31
  \end{tabbing}
\end{quote}
%
% Naturally, the system verifies the proof of the \QED\ step.  (Remember
% that it could verify \s22\ by itself.)
%     the QED step has nothing to do with \s22  -- KC
\ednote{KC}{Should be able to trim the text for this last level below}
Naturally, the \QED step is verified.  Although the system accepts
\OBVIOUS\ as the proof of \s31 (remember that it could verify \s22\ by
itself), we can provide more detail with yet another level
of proof.  We write this proof the way it would seem natural to a
person---by breaking it into two cases:
%
\begin{quote} \small
  \begin{tabbing}
    \THEOREM\ "\forall S : \forall f \in [S -> \SUBSET\ S] : \exists A \in \SUBSET\ S : \forall x \in S : f[x] \neq A" \kill
    \PROOF \kill
    \LSP \= \s11.\ \= \ASSUME \= "\NEW\ S", \kill
         \>        \>         \> "\NEW\ f \in [S -> \SUBSET\ S]" \kill
         \>        \> \PROVE "\exists A \in \SUBSET\ S : \forall x \in S : f[x] \neq A" \kill
         \>        \> \PROOF \kill
         \> \hspace{1em} \= \s21.\ \= \DEFINE "T == \{z \in S : z \notin f[z]\}" \kill
         \>        \> \s22.  \> "\forall x \in S : f[x] \neq T" \kill
         \>        \>  \hspace{1em} \= \s31.\ \= \ASSUME "\NEW\ x \in S" \PROVE "f[x] \neq T" \\
         \>        \>        \> \hspace{1em} \= \s41.\ \= \CASE "x \in T" \\
         \>        \>        \>        \> \s42.\ \> \CASE "x \notin T" \\
         \>        \>        \>        \> \s43.\ \> \QED \BY \s41, \s42
  \end{tabbing}
\end{quote}
The (omitted) proof of the \CASE\ statement \s41\ has as its goal
"f[x]\neq T" and has the additional usable fact $x\in T$ in its context.

\llnote{Minor edit below.}

We continue refining the proof in this way, stopping 
with an \OBVIOUS or \BY proof when a goal is
obvious enough for the back-end prover or for a human reader,
depending on who the proof is being written for. 
A \BY\ statement can guide the prover or the human reader
by listing helpful obvious consequences of known facts.  For example,
the proof of \s41\ might be "\BY\ x \notin f[x]".
%
The complete proof appears in Appendix~\ref{apx:cantor}.

This example illustrates how the proof language supports a
hierarchical, non-linear, and incremental development of proofs.  The
proof writer can work on the most problematic unproved steps first,
leaving the easier ones for later.  Finding that a step cannot be
proved (for example, because it is invalid) may require changing other
steps, making proofs of those other steps wasted effort.  We intend to
provide an interface to the \PM\ that will make it easy for the user
to indicate which proofs should be checked and will avoid
unnecessarily rechecking proofs.

The example also shows how already-proved facts are generally not made
usable, but are invoked explicitly in \BY\ proofs.  Global definitions
are also hidden by default and the user must explicitly make them
usable.  This makes proofs easier to read by telling the reader what
facts and definitions are being used to prove each step.  It also
helps constrain the search space for an automated back-end prover,
leading to more efficient verification.  Facts and definitions can be
switched between usable and hidden by \USE\ and \HIDE\ steps, which
have the same syntax as \BY. As noted above, omitting the label from a
step's starting token (for example, writing \s4 instead of \s42) makes
the fact it asserts usable.  This might be done for compactness at
the lowest levels of a proof.

% The Isar proof language~\cite{isar} is similarly explicit about
% usable assumptions, but the facility is local to a proof and does
% not persist into sub-proofs, causing repetition in sibling
% sub-proofs. In the \tlatwo proof language, the visibility of an
% assumption persists until hidden, so one \USE declaration at a
% suitable level is enough. (The assumption can be selectively hidden
% at lower levels if needed for performance reasons.)

% \llnote{The preceding stuff about Isar needs to be moved to the conclusion.}


The example also indicates how the current proof obligation at every
step of the proof is clear, having been written explicitly in a parent
assertion.  This clear structure comes at the cost of introducing many
levels of proof, which can be inconvenient.  One way of avoiding these
extra levels is by using an assertion of the form "\SUFFICES\ A",
which asserts that proving $A$ proves the current goal, and makes $A$
the new current goal in subsequent steps.  In our example proof, one
level in the proof of step \s22\ can be eliminated by writing the
proof as:
\begin{quote} \small
  \begin{tabbing}
    \THEOREM\ "\forall S : \forall f \in [S -> \SUBSET\ S] : \exists A \in \SUBSET\ S : \forall x \in S : f[x] \neq A" \kill
    \PROOF \kill
    \LSP \= \s11.\ \= \ASSUME \= "\NEW\ S", \kill
         \>        \>         \> "\NEW\ f \in [S -> \SUBSET\ S]" \kill
         \>        \> \PROVE "\exists A \in \SUBSET\ S : \forall x \in S : f[x] \neq A" \kill
         \>        \> \PROOF \kill
         \> \hspace{1em} \= \s21.\ \= \DEFINE "T == \{z \in S : z \notin f[z]\}" \kill
         \>        \> \s22.  \> "\forall x \in S : f[x] \neq T" \\
         \>        \>  \hspace{1em} \= \s31.\ \= \SUFFICES \ASSUME "\NEW\ x \in S" \PROVE "f[x] \neq T" \\
         \>        \>        \> \hspace{1em} \PROOF \OBVIOUS \\
         \>        \>        \> \s32.\ \> \CASE "x \in T" \\
         \>        \>        \> \s33.\ \> \CASE "x \notin T" \\
         \>        \>        \> \s34.\ \> \QED \BY \s32, \s33
  \end{tabbing}
\end{quote}
where the proofs of the \CASE\ steps are the same as before.  The
\SUFFICES\ statement changes the current goal of the level-3 proof to
$f[x]\neq T$ after adding a declaration of "x" and the usable fact "x
\in S" to the context. This way of proving a universally quantified
formula is sufficiently common that \tlatwo\ provides a \TAKE\
construct that allows the \SUFFICES\ assertion \s31 and its \OBVIOUS
proof to be written \mbox{\,$\TAKE\ x \in S$\,}. 

There is a similar construct, "\WITNESS\ f \in S" for proving an
existentially quantified goal $\exists x\in S: e$, which changes the
goal to "e[x := f]".
% the := notation for capture-avoiding substitution is universal
% enough, especially in the community reading this paper
For implicational goals "e => f", the construct $\HAVE\ e$ changes the
goal to $f$.  No other constructs in the \tlatwo proof language change
the form of the current goal. We advise that these constructs be used
only at the lowest levels of the proof, since the new goal they create
must be derived %% SM: commented out "algebraically"
instead of being available textually in
a parent assertion.  (As a check and an aid to the reader, one can at
any point insert a redundant \SUFFICES\ step that simply asserts the
current goal.)

The final \tlatwo\ proof construct is $\PICK\ x : e$, which introduces
a new symbol $x$ that satisfies $e$.  The goal of the proof of this
\PICK step is $\exists x : e$, and it changes the context of
subsequent steps by adding a declaration of "x" and the fact "e". 
%
A more formal summary of the language appears in
Appendix~\ref{apx}.

The semantics of a \tlatwo\ proof is independent of any back-end
prover. Different provers will have different notions of what
``follows easily'', so an \OBVIOUS\ proof may be verified by one
prover and not another.  In practice, many provers such as Isabelle
must be directed to use decision procedures or special tactics to
prove some assertions.  For this purpose, special standard modules
will contain dummy theorems for giving directives to the
\PM.  Using such a theorem (with a \USE\ step or \BY\ proof) will
% LL: What generated proof obligation?  And why filter something
%     rather than not putting it into proof obligations in the 
%     first place?
cause the \PM\ not to use it as a fact, 
but instead to generate special directives for back-end
provers.  It could even cause the \PM\ to use a different back-end
prover.  (The dummy theorem will assert a true fact that suggests the
purpose of the directive.)
For instance, using the theorem \emph{Arithmetic}
%% LL: Engineers have never heard of Peano Arithmetic, and we don't
%%     have so much space that we can squander it on such an example.
%%   \textit{PeanoArithmetic} might assert the existence of a set of
%%   natural numbers satisfying the axioms of Peano Arithmetic; a "\BY\
%%   \mathit{PeanoArithmetic}" proof 
%
might be interpreted as an instruction to use a decision procedure for
integers.

\llnote{I replaced ``natural numbers'' by ``integers'' above, which seems
to me what arithmetic reasoning is more likely to mean.}

We hope that almost all uses of this feature will leave the \tlatwo
proof independent of the back-end prover(s).  The proof will not have
to be changed if the \PM\ is reconfigured to replace one decision
procedure with a different one.

%%%% PLEASE DO NOT EDIT BELOW THIS LINE
\ifx\master\undefined
{\let\master\relax%%% -*- mode: LaTeX; TeX-master: "main.tex"; -*-

\ifx\master\undefined
\documentclass[a4paper]{easychair}
\usepackage{submission}
\begin{document}
{\let\master\relax %%% -*- mode: LaTeX; TeX-master: "main.tex"; -*-

\ifx\master\undefined
\documentclass[a4paper]{easychair}
\usepackage{submission}
\begin{document}
\fi
%%%% PLEASE DO NOT EDIT ABOVE THIS LINE

\title{A \tlaplus Proof System}

\titlerunning{A \tlaplus Proof System}

% \volumeinfo
% 	{P. Rudnicki, G. Sutcliffe} % editors
% 	{2}                         % number of editors
% 	{KEAPPA 2008}               % event
% 	{1}                         % volume
% 	{1}                         % issue
% 	{1}                         % starting page number


%  Alphabetically by surname
\author{
  Kaustuv Chaudhuri \\
  INRIA \\
  \and
  Damien Doligez \\
  INRIA \\
  \and
  Leslie Lamport \\
  Microsoft Research \\
  \and
  Stephan Merz \\
  INRIA \& Loria
}

\authorrunning{Chaudhuri, Doligez, Lamport, and Merz}

\maketitle

%%%% PLEASE DO NOT EDIT BELOW THIS LINE
\ifx\master\undefined
{\let\master\relax \input{rearmatter}}
\end{document}
\fi

% LocalWords:  tex Rudnicki Sutcliffe KEAPPA Kaustuv Chaudhuri INRIA Doligez
% LocalWords:  Merz Loria
}
\fi
%%%% PLEASE DO NOT EDIT ABOVE THIS LINE

\bibliographystyle{plain}
\bibliography{submission}

%%%% PLEASE DO NOT EDIT BELOW THIS LINE
\ifx\master\undefined
\end{document}
\fi

% LocalWords:  tex Paxos
}
\end{document}
\fi
          % sketch + examples

%%% -*- mode: LaTeX; TeX-master: "main.tex"; -*-

\ifx\master\undefined
\documentclass[a4paper]{easychair}
\usepackage{submission}
\begin{document}
{\let\master\relax%%% -*- mode: LaTeX; TeX-master: "main.tex"; -*-

\ifx\master\undefined
\documentclass[a4paper]{easychair}
\usepackage{submission}
\begin{document}
\fi
%%%% PLEASE DO NOT EDIT ABOVE THIS LINE

\title{A \tlaplus Proof System}

\titlerunning{A \tlaplus Proof System}

% \volumeinfo
% 	{P. Rudnicki, G. Sutcliffe} % editors
% 	{2}                         % number of editors
% 	{KEAPPA 2008}               % event
% 	{1}                         % volume
% 	{1}                         % issue
% 	{1}                         % starting page number


%  Alphabetically by surname
\author{
  Kaustuv Chaudhuri \\
  INRIA \\
  \and
  Damien Doligez \\
  INRIA \\
  \and
  Leslie Lamport \\
  Microsoft Research \\
  \and
  Stephan Merz \\
  INRIA \& Loria
}

\authorrunning{Chaudhuri, Doligez, Lamport, and Merz}

\maketitle

%%%% PLEASE DO NOT EDIT BELOW THIS LINE
\ifx\master\undefined
{\let\master\relax %%% -*- mode: LaTeX; TeX-master: "main.tex"; -*-

\ifx\master\undefined
\documentclass[a4paper]{easychair}
\usepackage{submission}
\begin{document}
{\let\master\relax \input{frontmatter}}
\fi
%%%% PLEASE DO NOT EDIT ABOVE THIS LINE

\bibliographystyle{plain}
\bibliography{submission}

%%%% PLEASE DO NOT EDIT BELOW THIS LINE
\ifx\master\undefined
\end{document}
\fi

% LocalWords:  tex Paxos
}
\end{document}
\fi

% LocalWords:  tex Rudnicki Sutcliffe KEAPPA Kaustuv Chaudhuri INRIA Doligez
% LocalWords:  Merz Loria
}
\fi
%%%% PLEASE DO NOT EDIT ABOVE THIS LINE

\section{Proof Obligations}
\label{sec:obligations}

The \PM\ computes a separate proof obligation for each % (non-omitted)
leaf proof and orchestrates the back-end provers to verify these
obligations.  Each obligation is independent and can be proved
individually.  If the system cannot verify an obligation within a
reasonable amount of time, the \PM\ reports a failure.  The user must
then determine if it failed because it depends on hidden facts or
definitions, or if the goal is too complex and needs to be refined
with another level of proof.  (Hiding facts or definitions might also
help to constrain the search space of the back-end provers.)

When the back-end provers fail to find a proof, the user will know
which obligation was unprovable---that is, she will be told the
obligation's usable context and goal and the leaf proof from which it
was generated.  We do not yet know if this will be sufficient in
practice or if the \PM will need to provide the user with more
information about why an obligation failed.  For example, many SAT and
SMT solvers produce counterexamples for an unsatisfiable formula that
can provide useful information.

% \llnote{I added the preceding paragraph in response to the following.}
% \ednote{SM}{Should we discuss a bit more what information we expect
%   the \PM and the backend to exchange (note the name of the workshop
%   \ldots)? The user's task would become easier if the back-end
%   reported why it failed. For example, SMT solvers compute a
%   counter-example to the proof obligation.}
% \ednote{KC}{I don't understand what is meant by:
%   \begin{quote}
%     ``\dots the user will need to be told what subgoal Isabelle failed on.''
%   \end{quote}
%   What is a ``subgoal''?}

% If only these separate proof obligations were verified, then an
% error in the way the \PM\ generated any single obligation could make
% the proof incorrect.  We avoid placing so much trust in the \PM\ by
% having it generate an additional \emph{structural} obligation and its
% proof.  The structural obligation asserts simply that the collection
% of all these other obligations implies the correctness of the theorem.
% By verifying the structural obligation's proof, the back-end prover
% asserts that the proofs of all the obligations generated by the \PM
% prove the statement of the theorem.  Assuming that the prover is
% sound, this implies that an error is possible only if the \PM\ made an
% error in stating the (essentially trivial) structural obligation, or
% it incorrectly translated the statement of the theorem into the
% prover's logic.
% %
%
% \llnote{Kaustuv replaced the preceding paragraph with one that says
% so many things that I don't agree with or don't understand that I
% have not attempted to incorporate it.  Instead, I've inserted it
% as a comment with my remarks.
% }
%
% \ednote{KC}{Needless to say, I dislike the previous paragraph very
%   much. My version, uncommented, with explanations below.
%   \begin{quote}
%     The mechanics of the \PM are not trusted. 
%     %
%     \ednote{LL}{What are the ``mechanics'' of the \PM.  If the \PM\
%       isn't trusted, then we don't trust any proof it verifies.
%       %
%       \ednote{KC}{Maybe I should have said ``proof obligation
%         generating procedure'' instead of ``mechanics''. The whole
%         point of certifying with Isabelle/\tlaplus is that we don't
%         have to trust whatever voodoo the PM does.}
%       %
%     }
%     %
%
%    Instead, the \PM proves that the obligations it computes for a
%    given proof will successfully lead to a proof of the goal.  Thus,
%    a bug in the \PM that causes it to emit incorrect obligations, or
%    miss important obligations, will not cause the user to believe her
%    \tlatwo proof is checked when indeed it is wrong.
%
%    \ednote{LL}{This is nonsense.  If the \PM\ produces TRUE as as the
%      goal, how does proving that the proof obligations ``lead to it''
%      (whatever that means) prove anything?
%      %
%      \ednote{KC}{Please \textit{reread} the paragraph, particularly
%        the end. We have said that the embedding of \tlaplus to
%        Isabelle/\tlaplus is trusted. If it performs an unsound
%        translation, such as translating \FALSE to \TRUE, then all bets
%        are off.
%        %
%      }
%      %
%    }
%
%    This sufficiency proof proves a \textit{structure theorem}, which
%    is an Isabelle/TLA+ meta-implication
%
%    \ednote{LL}{What is a meta-implication?
%      %
%      \ednote{KC}{An implication in the Isabelle meta-language. What is
%        a ``meta equality'', you may as well ask. You didn't have a
%        problem with that when Stephan suggested it.}
%    %
%    }
%
%    from the proof obligations to the goal of the proof, and is
%    rendered as an Isar proof script that is certified by
%    Isabelle/\tlaplus.  This proof is also statically computed,
%    requiing no proof search in its construction or certification.
%    The composition of the structure theorem and the proof
%    obligations is then a complete proof of the original goal,
%    certified by Isabelle/\tlaplus.
%  
%    \ednote{LL}{ In a previous email, you indicated that the proof of
%      what we were then calling the structure lemma does not contain
%      proofs of the individual obligations.  You now are contradicting
%      that.  Indeed, your 4-step description of how the structural
%      lemma was used did not show any proofs of obligations in the Isar
%      scripts sent to Isabelle.
%      %
%      \ednote{KC}{I fail to see any contradiction. This entire
%        paragraph depends on the individual proof obligations being
%        certified earlier. It could be made clearer, perhaps.}
%    %
%    }
%
%    Assuming that the Isabelle kernel is sound, and trusting the
%    embedding of \tlatwo into Isabelle/\tlaplus, the user may
%    therefore conclude that her proof is indeed correct.
%
%    \ednote{LL}{ So we're not willing to trust the PM to do the
%      horribly difficult task of making sure it sends every obligation
%      to the back-end to be checked, but we're willing to trust that
%      there are no errors in the trivial task of correctly embedding
%      \tlatwo\ in Isabelle (which includes both the translation to
%      Isabelle/\tlaplus and the axioms of Isabelle/\tlaplus?)
%      %
%      \ednote{KC}{Roughly, yes. We can't not trust the embedding, the
%        axiomatization, and the soundness of the Isabelle kernel.}
%    %
%    }
%    
%   \end{quote}
% }
%
% As an additional safeguard against possibly unsound provers, we
% require them to produce a proof whose correctness can be certified by
% Isabelle's kernel, the only ultimately trusted formalism.
% 
% \ednote{SM}{I added the last sentence of the preceding paragraph in
%   response to Leslie's request at the end of
%   Sect.~\ref{sec:obligations}.
% \llnote{I removed it because we explicitly say that we don't require it.}
% }
%
% \llnote{This is a question for Kaustuv.  The answer probably won't
% change what we've written.\\[.2em]
% %
% I still don't understand the structural obligation.  Kaustuv
% agreed that for a proof consisting of a sequence of steps ending with
% QED, all of whose proofs are omitted, the structural obligation would
% simply assert that the goal implies itself.  How does adding a BY
% proof of the QED step change the obligation?  How does adding a BY
% proof of one of the other steps change the obligation?  How does
% adding a non-leaf proof of one of these steps change the obligation?
% }

%%% The following two paragraphs is my attempt at reconciling
%%% the various bits of disagreement. -- KC

% The \PM will also mediate the \emph{certification} of the \tlatwo
% proof in a formal axiomatization of \tlatwo in a trusted logical
% framework, which in the current design is Isabelle/\tlaplus (described
% in section~\ref{sec:backend.isa}). Although the \PM is designed
% generically and can support other similar frameworks, for the rest of
% this paper we will limit our attention to Isabelle/\tlaplus.

% As each proof obligation is proved by back-end provers, the proofs are
% certified by Isabelle/\tlaplus---that is, the Isabelle kernel is
% convinced that the obligation is true. After all the obligations are
% certified, and assuming no sub-proof was omitted in the \tlatwo proof,
% the \PM will then write an Isar proof that Isabelle/\tlaplus will
% certify as a proof of the overall theorem. This proof uses a
% \emph{structure lemma}, which is an Isabelle meta-implication from the
% proof obligations to the overall theorem, and can be seen as a
% certificate of correctness of the proof-obligation generation
% procedure in the \PM. Once Isabelle certifies that the composition of
% the structure lemma and the individual proof obligations it had
% certified earlier is a proof of the overall theorem, we may conclude
% that the overall theorem is true assuming the embedding of \tlatwo
% into Isabelle/\tlaplus is sound. Note that none of the back-end
% provers, nor the proof obligation generating procedure in the \PM, are
% ultimately trusted.

%% And this is my attempt -- DD

The \PM will also mediate the \emph{certification} of the \tlatwo
proof in a formal axiomatization of \tlatwo in a trusted logical
framework, which in the current design is Isabelle/\tlaplus (described
in Section~\ref{sec:backend.isa}). Although the \PM is designed
generically and can support other similar frameworks, for the rest of
this paper we will limit our attention to Isabelle/\tlaplus.
%
Assuming that Isabelle/\tlaplus is sound, once it has certified a
theorem we know that an error is possible only if the \PM incorrectly
translated the statement of the theorem into Isabelle/\tlaplus.

After certifying each obligation, certification of the theorem itself
is achieved in two steps. First, the \PM generates a \emph{structure
  lemma} (and its Isabelle/\tlaplus proof) that states simply that the
collection of proof obligations implies the theorem.  Then, the \PM
generates a proof of the theorem that uses the already-certified
obligations and structure lemma.
%%% Do we need to make the following explicit ?
% Note that it is the certifier's job to check that they are already
% certified.  We don't depend on the \PM for that.
When the certifier accepts that proof, we get the assurance that the
translated version of the theorem is true, regardless of any errors
the \PM might have made.

%%%

Of course, we expect the \PM\ to be correct.  We now explain why it
should be by describing how it generates the proof obligations from
the proof of a theorem. (Remember that we are considering only \tlatwo
formulas with no temporal operators.)
%
% We use a slightly enhanced version of \tlatwo\ in which
% assumptions (\ASSUME\ clauses) can include definitions, and we write
% "hide h" to indicate that an assumption $h$ is hidden.  (However,
% remember that we are considering only \tlatwo\ formulas with constants
% and no temporal operators.)  If $\Gamma$ is a sequence of assumptions,
% we let "\obl{\G ||- e}" be an abbreviation for $\ASSUME\ \Gamma\
% \PROVE\ e$, or for $e$ if $\Gamma$ is empty.  
%
%%% It is misleading to say assertions are an enhanced form of TLA+
%%% because they are in the TLA+ *meta* logic.
%
A theorem in \tlatwo represents a closed \emph{assertion} in the
\tlatwo meta-logic of the form "\obl{\G ||- e}", where "\G" is a
\emph{context} containing all the declarations, definitions, facts
(previous assumptions or theorems) and the assumptions introduced in
the theorem using an \ASSUME clause (if present), and "e" is a \tlatwo
formula that is the \emph{goal} of the theorem.

\ednote{SM}{I side with Leslie here. Also note that ``existence of an
  Isabelle/\tlaplus proof'' is at least ambiguous. Assuming that the
  encoding of \tlaplus in Isabelle is complete (as it should be), this
  just means ``semantically true'', but is useless because we have no
  clue how to find this proof. For me, ``verifiable'' is constructive:
  we have a proof method (or back-end) that gives us the proof. Below
  is my suggestion. I don't feel a need to talk about hiding in any
  more detail at this point, but feel free to add it back in if you
  think otherwise.}

A closed assertion is said to be \textit{true} if $e$ is entailed by
$\G$ in the formal semantics of \tlaplus~\cite{lamport03tla}. It is
said to be \emph{verifiable} if we have a proof of $e$ from $\G$ in
Isabelle/\tlaplus. Because we assume Isabelle/\tlaplus to be sound,
any verifiable assertion is true. The \PM follows the structure of
the \tlatwo proof to refine the initial assertion into \emph{proof
  obligations} that
represent the individual steps of the \tlatwo proof. 
The proof is accepted by the \PM if all proof
obligations are verifiable, with the back-end provers generating the
necessary Isabelle/\tlaplus proofs. While traversing the \tlatwo
proof, the \PM also generates the structure lemma and its proof, which
is then certified by Isabelle/\tlaplus. Together with the proof
obligations certified earlier, Isabelle/\tlaplus then has a complete
certified proof of the overall obligation.
% \ednote{DD}{I edited the previous sentence to make it consistent with
% the new ``trust'' description.\\[.4em]
%SM: reverted to ``assertion'' and keeping ``proof obligation'' for the
%leaves because that's also the terminology of the appendix.}

A \emph{claim} is a sentence of the form "\pi:\obl{\G ||- e}", where
$\pi$ is a \tlatwo\ proof, representing the verification task that
$\pi$ is a proof of the assertion "\obl{\G ||- e}". The \PM
generates the proof obligations of a claim by recursively traversing its
proof, using its structure to transform the assertion. Each proof step
results in a modification of the current context or goal, and the
final \QED step proves the assertion obtained as a result of these
transformations. A claim
is \emph{meaningful} if this transformation is defined for every step of
the proof. (An example of a meaningless claim is one that involves a
\TAKE\ step whose assertion does not have a universally quantified
goal.) Precisely, every step defines a transformation,
written
% \begin{gather*} \small
$
  "\sigma.\,\tau: \obl{\G ||- e} --> \obl{\D ||- f}"
$,
% \end{gather*}
which states that the assertion "\obl{\G ||- e}" is changed to the
assertion "\obl{\D ||- f}" by the step "\sigma.\,\tau".

\llnote{I changed ``refined'' to ``changed'' above because I don't see
any refinement going on.  But I'm not adamant about that.\\[.3em]
Also, I think ``the assertion'' is going to confuse people because
we use ``assertion' in different ways.  It would be
better if we introduced the notion
of a \emph{current} assertion in a proof, so we could talk about a
current assertion here instead of just ``the assertion''.}

\ednote{KC}{It is always going to be a refinement because the truth of
  the ``thing'' to the right of "-->" implies the truth of that to the
  left.\\[1ex]
  I agree about confusing confusing the ``thing'' "\obl{\G ||- e}"
  with assertion steps. I had called them obligations from the start
  for \textit{this very reason} and was very surprised that you
  preferred the more confusing conflation. I am going back to my
  original terminology.\\[.4em]
SM: I don't quite understand the fuzz. Assertions are transformed, and
``obligations'' have been introduced as assertions at the leaves, so
why not leave it at that. A possible confusion may arise if it is not
clear to which assertion we refer at any given point, but I hope the
wording is now sufficiently clear.
}

The recursive generation of proof obligations for claims and
transformations is specified using inference rules, with the
interpretation that the proof obligations of the claim or
transformation at the conclusion of a rule is the union of the proof
obligations of those at the premises of the rule.  For example, the
following rule is applied to generate the proof obligations for a
claim "\pi:\obl{\G ||- e}" when $\pi$ is a sequence of $n$ steps, for
$n>1$.
%
\begin{gather*} \small
  \I{"\sigma_1.\,\tau_1 \ \ \sigma_2.\,\tau_2 \ \ \dotsb \ \ \sigma_n.\,\tau_n : \obl{\G ||- e}"}
    {"\sigma_1.\,\tau_1 : \obl{\G ||- e} --> \obl{\D ||- f}"
     &
     "\sigma_2.\,\tau_2 \ \ \dotsb \ \ \sigma_n.\,\tau_n : \obl{\D ||- f}"}
\end{gather*}
As an example of proof obligations generated by a transformation, here
is a rule for a step numbered with the begin-step level token $\s{n}$
that asserts "a" and has the proof $\pi$.
\begin{gather*} \small
  \I{"\s{n}.\ a\ \PROOF\ \pi : \obl{\G ||- e} --> \obl{\G, a ||- e}"}
    {"\pi : \obl{\G, \hide{\lnot e} ||- a}"}
\end{gather*}
The rule concludes that the result of this step is to add "a" to
the context, assuming that the sub-proof "\pi" is able to establish
it.
% (If the step-starting token were a label $\s{n}l$, then "a" would
% have been hidden in the result and "\s{n}l\DEF a" would have been
% added as a usable definition.)
%%   that's just a distraction here -- KC
% The proof obligations generated by this transformation are the same as
% those of the claim in the premise of the rule. 
%%   redundant as we have already said what a rule means
The input goal "e" is negated and added to the context hidden, written
"\hide{\lnot e}", in the premise; it can be added to the context
because the logic is classical, and can simplify subproofs which may
now appeal to "\lnot e" with \BY or \USE.  A complete set of such
rules is given in the appendix.

These rules generate proof obligations for any meaningful claim. The
\PM instructs the back end to prove the claims after
%\textit{filtering} all hidden assumptions from them, which amounts to
deleting hidden facts and replacing operator definitions with
declarations.
% ; these filtered obligations are consequently certified in
% Isabelle/\tlaplus. 
The \PM can then construct a proof of the overall assertion
that uses the (already certified) leaf obligations;
the following meta-theorem (whose proof appears in
appendix~\ref{apx:constraints}) justifies the correctness of this
approach.
%
\begin{thm}[Structural Soundness Theorem] \label{thm:meaning}
  % 
  If "\pi:\obl{\G ||- e}" is a meaningful claim and the proof
  obligations it generates all verifiable, then "\obl{\G ||- e}" is
  true.
\end{thm}

%%%% PLEASE DO NOT EDIT BELOW THIS LINE
\ifx\master\undefined
{\let\master\relax%%% -*- mode: LaTeX; TeX-master: "main.tex"; -*-

\ifx\master\undefined
\documentclass[a4paper]{easychair}
\usepackage{submission}
\begin{document}
{\let\master\relax %%% -*- mode: LaTeX; TeX-master: "main.tex"; -*-

\ifx\master\undefined
\documentclass[a4paper]{easychair}
\usepackage{submission}
\begin{document}
\fi
%%%% PLEASE DO NOT EDIT ABOVE THIS LINE

\title{A \tlaplus Proof System}

\titlerunning{A \tlaplus Proof System}

% \volumeinfo
% 	{P. Rudnicki, G. Sutcliffe} % editors
% 	{2}                         % number of editors
% 	{KEAPPA 2008}               % event
% 	{1}                         % volume
% 	{1}                         % issue
% 	{1}                         % starting page number


%  Alphabetically by surname
\author{
  Kaustuv Chaudhuri \\
  INRIA \\
  \and
  Damien Doligez \\
  INRIA \\
  \and
  Leslie Lamport \\
  Microsoft Research \\
  \and
  Stephan Merz \\
  INRIA \& Loria
}

\authorrunning{Chaudhuri, Doligez, Lamport, and Merz}

\maketitle

%%%% PLEASE DO NOT EDIT BELOW THIS LINE
\ifx\master\undefined
{\let\master\relax \input{rearmatter}}
\end{document}
\fi

% LocalWords:  tex Rudnicki Sutcliffe KEAPPA Kaustuv Chaudhuri INRIA Doligez
% LocalWords:  Merz Loria
}
\fi
%%%% PLEASE DO NOT EDIT ABOVE THIS LINE

\bibliographystyle{plain}
\bibliography{submission}

%%%% PLEASE DO NOT EDIT BELOW THIS LINE
\ifx\master\undefined
\end{document}
\fi

% LocalWords:  tex Paxos
}
\end{document}
\fi
             % high level overview of
                                % the PM

%%% -*- mode: LaTeX; TeX-master: "main.tex"; -*-

\ifx\master\undefined
\documentclass[a4paper]{easychair}
\usepackage{submission}
\begin{document}
{\let\master\relax%%% -*- mode: LaTeX; TeX-master: "main.tex"; -*-

\ifx\master\undefined
\documentclass[a4paper]{easychair}
\usepackage{submission}
\begin{document}
\fi
%%%% PLEASE DO NOT EDIT ABOVE THIS LINE

\title{A \tlaplus Proof System}

\titlerunning{A \tlaplus Proof System}

% \volumeinfo
% 	{P. Rudnicki, G. Sutcliffe} % editors
% 	{2}                         % number of editors
% 	{KEAPPA 2008}               % event
% 	{1}                         % volume
% 	{1}                         % issue
% 	{1}                         % starting page number


%  Alphabetically by surname
\author{
  Kaustuv Chaudhuri \\
  INRIA \\
  \and
  Damien Doligez \\
  INRIA \\
  \and
  Leslie Lamport \\
  Microsoft Research \\
  \and
  Stephan Merz \\
  INRIA \& Loria
}

\authorrunning{Chaudhuri, Doligez, Lamport, and Merz}

\maketitle

%%%% PLEASE DO NOT EDIT BELOW THIS LINE
\ifx\master\undefined
{\let\master\relax %%% -*- mode: LaTeX; TeX-master: "main.tex"; -*-

\ifx\master\undefined
\documentclass[a4paper]{easychair}
\usepackage{submission}
\begin{document}
{\let\master\relax \input{frontmatter}}
\fi
%%%% PLEASE DO NOT EDIT ABOVE THIS LINE

\bibliographystyle{plain}
\bibliography{submission}

%%%% PLEASE DO NOT EDIT BELOW THIS LINE
\ifx\master\undefined
\end{document}
\fi

% LocalWords:  tex Paxos
}
\end{document}
\fi

% LocalWords:  tex Rudnicki Sutcliffe KEAPPA Kaustuv Chaudhuri INRIA Doligez
% LocalWords:  Merz Loria
}
\fi
%%%% PLEASE DO NOT EDIT ABOVE THIS LINE


% \llnote{Section 4.2 makes sense only if the reader knows that we currently
%  plan to use Isabelle as the ``ultimate authority'' to check the complete proof.
% This needs to be stated somewhere above---perhaps in the place in Section 3
% that talks about the structural obligation.}
% 
\section{Verifying Proof Obligations}
\label{sec:backend}

% \llnote{I changed the title of this section so it covers Section 4.1 
% (whose title I also changed).}
%\llnote{It would be nice to add an introductory sentence or two here.}
% \ednote{DD}{This is rather redundant with the beginning of section 3
%   and the end of section 4.2}
% \llnote{I removed a bunch of redundant stuff here and added the
% following intro to this section to try to justify its structure.  The
% problem is that Section 4.1 is something of an orphan.  It belongs in
% a section about what the PM does besides generating
% obligations---namely, checking that a proof step is ``meaningful'',
% and translating the obligations to the language of the back-end
% provers.  But since we don't have much to say about this, it gets
% tucked into a section about expanding definitions.  I want to justify
% this by saying that expanding definitions is the only interesting
% thing the PM does other than generating proof obligations.
% I've sort of done that in this intro.  Kaustuv, please check that
% you agree with what I've written---and correct it if you don't.}

Once the \PM\ generates the primitive proof obligations, it must send
them to the back-end provers.  The one non-obvious part of doing this
is deciding whether definitions should be expanded by the \PM\ or by
the prover.  This is discussed in Section~\ref{sec:backend.pm}.
We then describe the state of our two current back-end provers,
Isabelle/\tlaplus\ and Zenon.

% they are sent
% to a back-end prover that will try to verify them with automated
% proof procedures and report success or failure.
% The failure case is reported back to the user who must determine
% whether it failed because the automated procedures are not powerful
% enough (in which case the user will add more details to the proof)
% or because something is wrong with the corresponding leaf proof.
% The structural obligation is also sent to the back-end prover in
% order to provide high assurance of the correctness of the proof.
% 
% In our design, Isabelle plays two roles: on the first hand
% Isabelle/\tlaplus{} (the Isabelle kernel with the \tlaplus axioms)
% is the certifier that provides the ultimate assurance of correctness
% by certifying the proofs provided by the automatic procedures,
% on the other hand Isabelle provides some of the automatic proof procedures
% that are used to verify the proof obligations.

\subsection{Expanding Definitions}
\label{sec:backend.pm}

Expansion of usable definitions cannot be left entirely to the
back-end prover.  The \PM itself must do it for two reasons:
\begin{icom}
\item It must check that the current goal has the right form for a
\TAKE, \WITNESS, or \HAVE\ step to be meaningful, and this can require
expanding definitions.

% \item Using TLA's modal operators such as prime (which we have been
% ignoring), we can define an operator $O$ such that a prover based on
% classical logic could verify an invalid formula if it contains $O$.
% All instances of such operators must be removed by expanding their
% definitions before a proof obligation is sent to the back-end prover.
% (These operators seldom occur in practice.)

\llnote{I rewrote the following.  It could use polishing.}

\item The encoding of TLA+ in a back-end prover would be unsound if a
modal operator like prime ($'$) were represented as an operator.
Hence, encoding a definition like $O(x)\DEF x'$ as a definition in the
prover's logic would be unsound.  All instances of such operators must
be removed by expanding their definitions before a proof obligation is
sent to the back-end prover.  (These operators seldom occur in
practice.)

\end{icom}
%
% \llnote{Kaustuv: Before you object to the preceding sentence, remember
% that \ENABLED\ is not a defined operator.  Since we're not discussing
% \ENABLED\ here, it would be an unnecessary distraction to say that
% the PM must expand~it.}
%
Another reason for the \PM to handle definition expansion is that the
Isabelle/\tlaplus object logic does not provide a direct encoding of
definitions made within proofs.  The \PM therefore now performs all
definition expansions.  We plan to reduce the amount of trusted code
in the \PM\ by lambda-lifting all definitions out of each proof
obligation and using Isabelle meta equality ($\equiv$) with the
\texttt{unfolding} tactic before instructing Isabelle to search for a
proof or to certify a given proof.

% \ednote{SM}{I still don't understand the remark about definitions in
%   Isabelle/\tlaplus. Are you saying that the current implementation of
%   the \PM\ doesn't generate corresponding Isabelle definitions? A
%   constant-level operator definition
%   \[
%     Op(x,y) \DEF exp
%   \]
%   can simply be written like this in Isabelle. ``Lambda-lifting'' is
%   done by the Isabelle toplevel. Level-1 or 2 \tlaplus definitions
%   require a little more care, but we omit them here.}
% 
% \begin{quote}
%   \ednote{KC}{I assume you mean "\And x\ y.\ \mathit{Op}(x, y) \equiv
%     \mathit{exp}", and possibly even "\And (x :: c).\ \And (y :: c).\
%     \mathit{Op}(x, y) \equiv \mathit{exp}" because Isabelle can
%     otherwise infer an overly general type for "\mathit{Op}". Even so,
%     being Pure syntax, such a method only applies to the outermost
%     operator definitions. The only way to use them for inner
%     definitions (\ie, those inside \kwd{LET}-expressions) for the time
%     being is to lift them out.}
% \end{quote}
% 
% \llnote{Kaustuv told me that this kind of definition mechanism, which
%   Isabelle provides, was not implemented as part of Isabelle/\tlaplus\
%   for some reason that I forget.  Stephan seems to be saying that
%   Kaustuv was mistaken, and that he should not be expanding
%   definitions when sending things to Isabelle (except of course for
%   non-Leibniz operators).  Will the two of you please work this out by
%   email or telephone and let us know what the story is.}


\subsection{Isabelle/\tlaplus}
\label{sec:backend.isa}

The core of \tlatwo is being encoded as a new object logic
Isabelle/\tlaplus in the proof assistant
Isabelle~\cite{paulson:isabelle}.
% This encoding serves two distinct
% purposes in the overall project. First, we rely on Isabelle's trusted
% kernel to certify the overall correctness of a proof by checking the
% structural obligation described in Section~\ref{sec:obligations}.
% Second, Isabelle can be used as one of the back-end provers to
% establish leaf steps of a proof.
% 
% \ednote{SM}{I have commented out a remark on the two roles of
%   Isabelle, which Leslie added to the introduction of Sect.~4. If you
%   think it should nevertheless be repeated here, uncomment it.}
% 
One of Isabelle's distinctive features that similar proof assistants
such as Coq~\cite{coq} or HOL~\cite{gordon:hol,harrison:hol} lack is
genericity with respect to different logics. The base system
Isabelle/Pure provides the trusted kernel and a framework in which the
syntax and proof rules of object logics can be defined. We have chosen
to encode \tlatwo as a separate object logic rather than add it on top
of one of the existing logics (such as ZF or HOL). This simplifies the
translation and makes it easier to interpret the error messages when
Isabelle fails to prove obligations. A strongly typed logic such as
HOL would have been unsuitable for representing \tlatwo, which is
untyped. Isabelle/ZF might seem like a natural choice, but differences
between the way it and \tlaplus{} define functions and tuples would
have made the encoding awkward and would have prevented us from
reusing existing theories. Fortunately, the genericity of Isabelle
helped us not only to define the new logic, but also to instantiate the
main automated proof methods, including rewriting, resolution- and
tableau provers, and case-based and inductive reasoning.
Adding support for more specialized reasoning tools such as
proof-producing SAT solvers~\cite{fontaine:automation} or SMT solvers
such as haRVey~\cite{deharbe:decision} will be similarly helped by
existing generic interfaces.

% \ednote{LL}{I did some rewording of the preceding paragraph for
%   readability. I don't think I changed what Stephan was trying to say.
%   However, do COQ and HOL really lack genericity, as I think Stephan
%   claimed?\\[.3em]
% SM: In the sense that they impose a fixed logic (calculus of
% constructions or simply-typed higher order logic). This is not to
% say that these logics are not sufficiently expressive to define
% \tlatwo{} but typically one would have to define new operators for
% the Boolean connectives etc. instead of reusing those that already
% exist. Isabelle/Pure only has implication and quantification, so
% there is not much to redefine. Please tell me if I should make this
% clearer above. (I assumed that the members of the community
% organizing this workshop are familiar with this problem.)}
% 
% \ednote{SM}{ I could elaborate a little on the overall structure of
%   the encoding, in particular the absence of types, but I wonder if
%   this is interesting enough.}
% 
% \llnote{It looks OK to me.  However, assuming that our 10-page limit
%   is not supposed to count the reference list, we do have a reasonable
%   amount of extra space.  But rather than talking about the overall
%   structure, is it possible to talk about one small aspect of the
%   encoding in enough detail to give a flavor of what the whole thing
%   looks like?}
% 
% \ednote{KC}{It is 10 pages including the bibliography and excluding
%   the appendix.}

The current encoding supports only a core subset of \tlatwo, including
propositional and first-order logic, elementary set theory, functions,
and the construction of natural numbers.  Support for arithmetic,
strings, tuples, sequences, and records is now being added; support
for the modal part of \tlatwo{} (variables, priming, and temporal
logic) will be added later.  Nevertheless, the existing fragment can
already be used to test the interaction of the \PM\ with Isabelle and
other back-end provers.  As explained above, Isabelle/\tlaplus is used
both as a back-end prover and to certify proof scripts produced by
other back-end provers such as Zenon.  If it turns out to be
necessary, we will enable the user to invoke 
% 
% LL: I eliminated the following prose, which says nothing that
%     hasn't already been explained
%
%    When the \PM generates a proof obligation, it can pass it to a
%    prover that produces a full proof as an Isar proof script (such as
%    Zenon, described in Section~\ref{sec:backend.zenon} below).
%    Isabelle/\tlaplus processes the proof script and generates a
%    theorem that is certified by the kernel and can be used in
%    subsequent steps.  Alternatively,
% 
one of Isabelle's automated proof methods (such as \texttt{auto} or
\texttt{blast}) by using a dummy theorem, as explained at the end of
Section~\ref{sec:proof-language.lang}.  If the method succeeds, one
again obtains an Isabelle theorem.  Of course,
%
Isabelle/\tlaplus can also be used independently of the \PM, which is
helpful when debugging tactics.
% We plan to validate the interaction by porting some existing case
% studies to Isabelle/\tlaplus, such as the proof of the Disk Paxos
% consensus algorithm~\cite{gafni:disk-paxos} and a verification of
% distributed clock synchronization
% algorithms~\cite{barsotti:verification}.
\llnote{I removed a sentence about porting existing case studies to
Isabelle/\tlaplus\ from the end of the previous paragraph, and I
replaced it with the ``helpful when debugging tactics.''.  I expect
Stephan to rewrite this if he can find something more accurate or
precise to say instead.  The problem with the original is it raised
the question, why not rewrite the (presumably Isar) proofs in \tlatwo\
and run them through the \PM\ rather than directly in
Isabelle/\tlaplus.  I presume we are planning to do this.  Should we
say so?\\[.4em]
%
SM: Indeed, I wanted to say that we plan to port the existing examples
to \tlatwo, using the \PM. The preceding sentence about independent
use of Isabelle/\tlaplus{} indeed obscured the meaning. Anyway, it is
future work and should, if at all, go to the conclusion.}


\subsection{Zenon}
\label{sec:backend.zenon}

Zenon~\cite{bonichon07lpar} is a tableau prover for classical
first-order logic with equality that was initially designed to output
formal proofs checkable by Coq~\cite{coq}.  We have extended Zenon to
output Isar proof scripts for Isabelle/\tlaplus theorems, and the \PM
uses Zenon as a back-end prover, shipping the proofs it produces to
Isabelle for certification.  We have also extended Zenon with direct support
for the \tlatwo logic, including definitions and rules about sets and
functions.  Adding support in the form of rules (instead of axioms) is
necessary because some rules are not expressible as first-order
axioms, notably the rules about the set constructs:
\begin{gather*} \small
  \I[\textit{subsetOf}]{e\in\{x\in S:P\}}{e\in S & P[x:=e]}
  \qquad
  \I[\textit{setOfAll}]{e\in\{d:x\in S\}}{\exists y\in S:e=d[x:=y]}
\end{gather*}
Even for the rules that are expressible as first-order axioms, adding them
as rules makes the proof search procedure much more efficient in practice.
The most important example is extensionality: when set extensionality and
function extensionality are
added as axioms, they apply to every equality deduced by the system,
and pollute the search space with large numbers of irrelevant formulas.
By adding them as rules instead, we can use heuristics to apply them only
in cases where they have some chance of being useful.

Adding support for arithmetic, strings, tuples, sequences, and records
will be done in parallel with the corresponding work on
Isabelle/\tlaplus, to ensure that Zenon will produce proofs that
Isabelle/\tlaplus\ will be able to certify.  Adding temporal logic
will be left to future research.
%
% \llnote{I added to the preceding paragraph an explanation of why
% adding support for those things will be done in parallel with
% the Isabelle work.  Damien, please check that this explanation is
% correct.}
%
We also plan to interface Zenon with Isabelle, so it can be called by a
special Isabelle tactic the same way other tools
%% SM: removed reference to haRVey because the haRVey interface is not
%% currently part of the Isabelle distribution.
are.
This
%
% LL: People are not concerned with how much others have to work
% 
%    represents more implementation work but 
% 
will simplify the \PM\ by giving it a uniform interface to the
back-end provers.  It will also allow using Zenon as an Isabelle tactic
independently of \tlaplus{}.


%%%% PLEASE DO NOT EDIT BELOW THIS LINE
\ifx\master\undefined
{\let\master\relax%%% -*- mode: LaTeX; TeX-master: "main.tex"; -*-

\ifx\master\undefined
\documentclass[a4paper]{easychair}
\usepackage{submission}
\begin{document}
{\let\master\relax %%% -*- mode: LaTeX; TeX-master: "main.tex"; -*-

\ifx\master\undefined
\documentclass[a4paper]{easychair}
\usepackage{submission}
\begin{document}
\fi
%%%% PLEASE DO NOT EDIT ABOVE THIS LINE

\title{A \tlaplus Proof System}

\titlerunning{A \tlaplus Proof System}

% \volumeinfo
% 	{P. Rudnicki, G. Sutcliffe} % editors
% 	{2}                         % number of editors
% 	{KEAPPA 2008}               % event
% 	{1}                         % volume
% 	{1}                         % issue
% 	{1}                         % starting page number


%  Alphabetically by surname
\author{
  Kaustuv Chaudhuri \\
  INRIA \\
  \and
  Damien Doligez \\
  INRIA \\
  \and
  Leslie Lamport \\
  Microsoft Research \\
  \and
  Stephan Merz \\
  INRIA \& Loria
}

\authorrunning{Chaudhuri, Doligez, Lamport, and Merz}

\maketitle

%%%% PLEASE DO NOT EDIT BELOW THIS LINE
\ifx\master\undefined
{\let\master\relax \input{rearmatter}}
\end{document}
\fi

% LocalWords:  tex Rudnicki Sutcliffe KEAPPA Kaustuv Chaudhuri INRIA Doligez
% LocalWords:  Merz Loria
}
\fi
%%%% PLEASE DO NOT EDIT ABOVE THIS LINE

\bibliographystyle{plain}
\bibliography{submission}

%%%% PLEASE DO NOT EDIT BELOW THIS LINE
\ifx\master\undefined
\end{document}
\fi

% LocalWords:  tex Paxos
}
\end{document}
\fi

% LocalWords:  tex reasoners sequents Zenon Isar HOL ZF minimizes TLA Paxos SMT
% LocalWords:  Charron Bost Schiper haRVey zenon
                 % Zenon back-end
%                                % Calling Isabelle
%
%%% -*- mode: LaTeX; TeX-master: "main.tex"; -*-

\ifx\master\undefined
\documentclass[a4paper]{easychair}
\usepackage{submission}
\begin{document}
{\let\master\relax%%% -*- mode: LaTeX; TeX-master: "main.tex"; -*-

\ifx\master\undefined
\documentclass[a4paper]{easychair}
\usepackage{submission}
\begin{document}
\fi
%%%% PLEASE DO NOT EDIT ABOVE THIS LINE

\title{A \tlaplus Proof System}

\titlerunning{A \tlaplus Proof System}

% \volumeinfo
% 	{P. Rudnicki, G. Sutcliffe} % editors
% 	{2}                         % number of editors
% 	{KEAPPA 2008}               % event
% 	{1}                         % volume
% 	{1}                         % issue
% 	{1}                         % starting page number


%  Alphabetically by surname
\author{
  Kaustuv Chaudhuri \\
  INRIA \\
  \and
  Damien Doligez \\
  INRIA \\
  \and
  Leslie Lamport \\
  Microsoft Research \\
  \and
  Stephan Merz \\
  INRIA \& Loria
}

\authorrunning{Chaudhuri, Doligez, Lamport, and Merz}

\maketitle

%%%% PLEASE DO NOT EDIT BELOW THIS LINE
\ifx\master\undefined
{\let\master\relax %%% -*- mode: LaTeX; TeX-master: "main.tex"; -*-

\ifx\master\undefined
\documentclass[a4paper]{easychair}
\usepackage{submission}
\begin{document}
{\let\master\relax \input{frontmatter}}
\fi
%%%% PLEASE DO NOT EDIT ABOVE THIS LINE

\bibliographystyle{plain}
\bibliography{submission}

%%%% PLEASE DO NOT EDIT BELOW THIS LINE
\ifx\master\undefined
\end{document}
\fi

% LocalWords:  tex Paxos
}
\end{document}
\fi

% LocalWords:  tex Rudnicki Sutcliffe KEAPPA Kaustuv Chaudhuri INRIA Doligez
% LocalWords:  Merz Loria
}
\fi
%%%% PLEASE DO NOT EDIT ABOVE THIS LINE

\section{Conclusions and Future Work}
\label{sec:conclusions}

We have presented a hierarchically structured proof language for
\tlaplus.  It has several important features that help in managing the
complexity of proofs.  The hierarchical structure means that changes
made at any level of a proof are contained inside that level, which
helps construct and maintain proofs.  Leaf proofs can be omitted and
the resulting incomplete proof can be checked.  This allows different parts
of the proof to be written separately, in a non-linear fashion.  
%%
%%  LL: I deleted 
% 
%    All other proof systems that we know of enforce a linear proof
%    style in which steps that have not yet been proved can be used
%    only if explicitly added as hypotheses.
% 
% Because the \emph{cheat} operator of Isar makes it false.
% 
The more traditional linear proof style, in which steps that have not
yet been proved can be used only if explicitly added as hypotheses,
encourages proofs that use many separate lemmas.  Such proofs lack the
coherent structure of a single hierarchical proof.

The proof language lets the user freely and repeatedly make facts and
definitions usable or hidden.  Explicitly stating what is being used
to prove each step makes the proof easier for a human to understand.
It also aids a back-end prover by limiting its search for a proof
to ones that use only necessary facts.

% \llnote{I added a citation to the Mizar web page.  Would someone please
% try to find out if there's a better reference?\\[.4em]
% SM: Reference added (it's pretty old, but I couldn't find a better one).}

There are other declarative proof languages that to some extent are
similar to \tlatwo.  Mizar~\cite{rudnicki:mizar} and Isar~\cite{isar} encourage
a traditional natural deduction style in which what has been proved
comes at the end rather than being stated as the goal of a lower-level
proof.  The Focal Proof Language~\cite{focal} is essentially a subset
of the \tlatwo\ proof language.  Our experience with hierarchical
proofs in Focal provides additional confidence in the attractiveness
of our approach.  We know of no declarative proof language that has as
flexible a method of using and hiding facts and definitions as
\tlatwo.

The \PM\ transforms a proof into a collection of proof obligations to
be verified by a back-end prover.  Its current version handles proofs
of theorems in the non-temporal fragment of \tlaplus\ that do not
involve module instantiation (importing of modules with substitution).
Even with this limitation, the system can be useful for many
engineering applications.  We are therefore concentrating on making
the \PM\ and its back-end provers handle this fragment of \tlaplus\
effectively before extending them to the complete language.  The major
work that remains to be done on this is to complete the Zenon and
Isabelle inference rules for reasoning about the built-in constant
operators of \tlaplus.  There are also a few non-temporal aspects of
the \tlatwo\ language that the \PM\ does not yet handle, such as
subexpression naming.  We also expect to extend the \PM\ to support
additional back-end provers, including decision procedures for
arithmetic and for propositional temporal logic.

%% I commented that paragraph because it is almost entirely redundant
%% with what has been said at several points in the paper and is not
%% really "concluding" anything. A reader who hasn't read the rest of
%% the paper will get little out ot it. Plus it lets the neighbouring
%% paragraphs flow into each other much better. -- KC
%%
%% LL: I sympathize with the sentiment, often omitting concluding 
%%     sections myself.  However, as was pointed out by someone (you?),
%%     one purpose of a conclusion is to summarize what we've said, which
%%     we need to do because at many (most?) reviewers will not read more 
%%     than the intro and the conclusion.  The first sentence repeats what
%%     has been said.  The second states two facts that appeared in
%%     different places.  I believe every other sentence says something
%%     not stated earlier.


% \llnote{Is the last sentence accurate?  In particular, if the decision
%   procedure will be called through Isabelle, does it constitute an
%   additional prover?  Do we need to clarify this?}
% 
% \ednote{DD}{I think it does count as an additional prover. Whether it
%   is called through Isabelle is an implementation detail, so we can
%   leave this sentence as it is.}
% 
% \ednote{SM}{I agree.}

We are satisfied with the proof language and do not anticipate that
any major changes will be needed.  We do expect some minor tuning as
we get more experience using it.  (For example, we are not sure
whether local definitions should be usable by default.)  A graphical
user interface is being planned for the \tlaplus\ tools, including the
\PM. It will support the non-linear development of proofs that the
language and the proof system allow.

%%%% PLEASE DO NOT EDIT BELOW THIS LINE
\ifx\master\undefined
{\let\master\relax%%% -*- mode: LaTeX; TeX-master: "main.tex"; -*-

\ifx\master\undefined
\documentclass[a4paper]{easychair}
\usepackage{submission}
\begin{document}
{\let\master\relax %%% -*- mode: LaTeX; TeX-master: "main.tex"; -*-

\ifx\master\undefined
\documentclass[a4paper]{easychair}
\usepackage{submission}
\begin{document}
\fi
%%%% PLEASE DO NOT EDIT ABOVE THIS LINE

\title{A \tlaplus Proof System}

\titlerunning{A \tlaplus Proof System}

% \volumeinfo
% 	{P. Rudnicki, G. Sutcliffe} % editors
% 	{2}                         % number of editors
% 	{KEAPPA 2008}               % event
% 	{1}                         % volume
% 	{1}                         % issue
% 	{1}                         % starting page number


%  Alphabetically by surname
\author{
  Kaustuv Chaudhuri \\
  INRIA \\
  \and
  Damien Doligez \\
  INRIA \\
  \and
  Leslie Lamport \\
  Microsoft Research \\
  \and
  Stephan Merz \\
  INRIA \& Loria
}

\authorrunning{Chaudhuri, Doligez, Lamport, and Merz}

\maketitle

%%%% PLEASE DO NOT EDIT BELOW THIS LINE
\ifx\master\undefined
{\let\master\relax \input{rearmatter}}
\end{document}
\fi

% LocalWords:  tex Rudnicki Sutcliffe KEAPPA Kaustuv Chaudhuri INRIA Doligez
% LocalWords:  Merz Loria
}
\fi
%%%% PLEASE DO NOT EDIT ABOVE THIS LINE

\bibliographystyle{plain}
\bibliography{submission}

%%%% PLEASE DO NOT EDIT BELOW THIS LINE
\ifx\master\undefined
\end{document}
\fi

% LocalWords:  tex Paxos
}
\end{document}
\fi
              % 
%
%%% -*- mode: LaTeX; TeX-master: "main.tex"; -*-

\ifx\master\undefined
\documentclass[a4paper]{easychair}
\usepackage{submission}
\begin{document}
{\let\master\relax %%% -*- mode: LaTeX; TeX-master: "main.tex"; -*-

\ifx\master\undefined
\documentclass[a4paper]{easychair}
\usepackage{submission}
\begin{document}
\fi
%%%% PLEASE DO NOT EDIT ABOVE THIS LINE

\title{A \tlaplus Proof System}

\titlerunning{A \tlaplus Proof System}

% \volumeinfo
% 	{P. Rudnicki, G. Sutcliffe} % editors
% 	{2}                         % number of editors
% 	{KEAPPA 2008}               % event
% 	{1}                         % volume
% 	{1}                         % issue
% 	{1}                         % starting page number


%  Alphabetically by surname
\author{
  Kaustuv Chaudhuri \\
  INRIA \\
  \and
  Damien Doligez \\
  INRIA \\
  \and
  Leslie Lamport \\
  Microsoft Research \\
  \and
  Stephan Merz \\
  INRIA \& Loria
}

\authorrunning{Chaudhuri, Doligez, Lamport, and Merz}

\maketitle

%%%% PLEASE DO NOT EDIT BELOW THIS LINE
\ifx\master\undefined
{\let\master\relax %%% -*- mode: LaTeX; TeX-master: "main.tex"; -*-

\ifx\master\undefined
\documentclass[a4paper]{easychair}
\usepackage{submission}
\begin{document}
{\let\master\relax \input{frontmatter}}
\fi
%%%% PLEASE DO NOT EDIT ABOVE THIS LINE

\bibliographystyle{plain}
\bibliography{submission}

%%%% PLEASE DO NOT EDIT BELOW THIS LINE
\ifx\master\undefined
\end{document}
\fi

% LocalWords:  tex Paxos
}
\end{document}
\fi

% LocalWords:  tex Rudnicki Sutcliffe KEAPPA Kaustuv Chaudhuri INRIA Doligez
% LocalWords:  Merz Loria
}
\fi
%%%% PLEASE DO NOT EDIT ABOVE THIS LINE

\bibliographystyle{plain}
\bibliography{submission}

%%%% PLEASE DO NOT EDIT BELOW THIS LINE
\ifx\master\undefined
\end{document}
\fi

% LocalWords:  tex Paxos
              % bibliography

\clearpage

\ifdraft\else
\setcounter{page}{1}
\renewcommand{\thepage}{(\roman{page})}
\fi

%%% -*- mode: LaTeX; TeX-master: "main.tex"; -*-

\ifx\master\undefined
\documentclass[a4paper]{easychair}
\usepackage{submission}
\begin{document}
{\let\master\relax%%% -*- mode: LaTeX; TeX-master: "main.tex"; -*-

\ifx\master\undefined
\documentclass[a4paper]{easychair}
\usepackage{submission}
\begin{document}
\fi
%%%% PLEASE DO NOT EDIT ABOVE THIS LINE

\title{A \tlaplus Proof System}

\titlerunning{A \tlaplus Proof System}

% \volumeinfo
% 	{P. Rudnicki, G. Sutcliffe} % editors
% 	{2}                         % number of editors
% 	{KEAPPA 2008}               % event
% 	{1}                         % volume
% 	{1}                         % issue
% 	{1}                         % starting page number


%  Alphabetically by surname
\author{
  Kaustuv Chaudhuri \\
  INRIA \\
  \and
  Damien Doligez \\
  INRIA \\
  \and
  Leslie Lamport \\
  Microsoft Research \\
  \and
  Stephan Merz \\
  INRIA \& Loria
}

\authorrunning{Chaudhuri, Doligez, Lamport, and Merz}

\maketitle

%%%% PLEASE DO NOT EDIT BELOW THIS LINE
\ifx\master\undefined
{\let\master\relax %%% -*- mode: LaTeX; TeX-master: "main.tex"; -*-

\ifx\master\undefined
\documentclass[a4paper]{easychair}
\usepackage{submission}
\begin{document}
{\let\master\relax \input{frontmatter}}
\fi
%%%% PLEASE DO NOT EDIT ABOVE THIS LINE

\bibliographystyle{plain}
\bibliography{submission}

%%%% PLEASE DO NOT EDIT BELOW THIS LINE
\ifx\master\undefined
\end{document}
\fi

% LocalWords:  tex Paxos
}
\end{document}
\fi

% LocalWords:  tex Rudnicki Sutcliffe KEAPPA Kaustuv Chaudhuri INRIA Doligez
% LocalWords:  Merz Loria
}
\fi
%%%% PLEASE DO NOT EDIT ABOVE THIS LINE

\appendix

% \def\nil{\bullet}
\def\nil{%
  \begin{picture}(0,1)(0,-3)%
    \circle*{2}
  \end{picture}%
}

The appendices are provided as an aid to the Programme Committee and
the reviewers of the submission. They contain a level of detail that
may be inappropriate for a final version, and any part of it that is
found to be invaluable in understanding the body of the paper will be
incorporated into the body.


\section{Details of the \PM}
\label{apx}

\ednote{KC}{Unblobbify the definition paragraphs in this page.}

\llnote{On a quick reading, this looks OK.  However, I think it's longer
than it needs to be.  Just because it isn't included in the page count
doesn't mean we shouldn't be as concise as possible.  Let's see if we
can cut unnecessary material.\\[.3em]
Also, we need to be consistent
in whether we write \emph{Def.} or \emph{Defn.} or \emph{Definition}.
In the main body of the text, I have eliminated such abbreviations
(e.g., using \emph{Section} instead of \emph{Sec.}).  But they're
OK in the appendix.
}

We shall now give a somewhat formal declarative specification of the
\PM and prove the key theorem~\ref{thm:meaning}. First, a quick
summary of the abstract syntax of \tlatwo proofs, ignoring the
stylistic aspects of their concrete representation.

\sdef{lev}#1{_{\s{#1}}}

\begin{defn}[\tlatwo Proof Language] \label{defn:proof-language}
  %
  % All constructs in the \tlatwo proof language have an associated
  % \emph{level}, which is a natural number representing the nesting
  % depth of the proof; a construct "\chi" of level "n" is written as
  % "\chi lev n". 
  \tlatwo \emph{proofs}, \emph{non-leaf proofs}, \emph{proof steps}
  and \emph{begin-step} tokens have the following syntax, where "n"
  ranges over natural numbers, "l" over labels, "e" over expressions,
  "\Phi" over lists of expressions, "o" over operator definitions,
  "\P" over sets of operator names, "\vec \beta" over lists of binders
  (\ie, constructs of the form "x" and "x \in e" used to build
  quantified expressions), and "\alpha" over expressions or \ASSUME
  \ldots \PROVE forms.

  \begin{quote} \itshape
    \begin{tabbing}
      (Proofs) \hspace{4.5em}  \= "\pi" \LSP \= "::=" \ \= "\OBVIOUS OR \OMITTED OR \BY\ \Phi\ \DEFS\ \P OR \Pi" \\
      (Non-leaf proofs) \> "\Pi" \> "::=" \> "\sigma_n.\ \QED\ \PROOF\ \pi" \\
                   \>             \> "\ \ \ |"  \> "\sigma_n.\ \tau\quad \Pi" \\
      (Proof steps) \> "\tau" \> "::=" \> "\USE\ \Phi\ \DEFS\ \P OR \HIDE\ \Phi\ \DEFS\ \P OR \DEFINE\ o" \\
                    \>          \> "\ \ \ |" \> "\HAVE\ e OR \TAKE\ \vec \beta OR \WITNESS\ \Phi" \\
                    \>          \> "\ \ \ |" \> "\alpha\ \PROOF\ \pi OR \SUFFICES\ \alpha\ \PROOF\ \pi 
                                                    OR \PICK\ \vec \beta : e\ \PROOF\ \pi" \\
      (Begin-step tokens) \> "\sigma_n" \> "::=" \> "\s n OR \s n l"
    \end{tabbing}
  \end{quote}
  A proof that is not a non-leaf proof is called a \emph{leaf
    proof}. The level numbers that occur in the begin-step tokens of
  the proof of a step (the "\pi" in "\alpha\ \PROOF\ \pi", \etc.) must be strictly higher than those of the step
  itself.
\end{defn}

\subsection{The Meta-Language}
\label{apx:context}

The \PM uses proofs in the \tlatwo proof language
(Def.~\ref{defn:proof-language}) to manipulate constructs in the
meta-language of \tlatwo. Although this meta-language naturally has no
representation in \tlatwo itself, we define its syntax formally as
follows.

\begin{defn}[Meta-Language] \label{defn:syntax}
  %
  The \tlatwo meta-language consists of \emph{assertions},
  \emph{assumptions} and \emph{definables} with the following syntax,
  where "e" ranges over \tlatwo expressions, "x" and "o" over \tlatwo
  identifiers, and "\vec x" over lists of \tlatwo identifiers.
  %
  \begin{quote}
    \begin{tabbing}
      (Assertions) \SP \= "\phi" \SP \LSP \= "::=" \ \= "\obl{h_1, \dotsc, h_n ||- e}" \` ("n \ge 0") \\
      (Assumptions) \> "h" \> "::=" \> "\NEW x OR o \DEF \delta OR \phi OR \hide{o \DEF \delta} OR \hide{\phi}" \\
      (Definables) \> "\delta" \> "::=" \> "\phi OR \LAMBDA\ \vec x : e"
    \end{tabbing}
  \end{quote}
  %
  The expression after "||-" in an assertion is called its
  \emph{goal}. An assumption written inside square brackets
  "\hide{\ }" is said to be \emph{hidden}; otherwise it is
  \emph{usable}. An assumption "h" can be \emph{made usable}, written
  "\unhide h", by removing the brackets surrounding it (if any).
  %
  A list of assumptions is called a \emph{context}, with the empty
  context written as "\nil"; we let "\G", "\D" and "\W" range over
  contexts, with "\G, \D" standing for the context that is the
  concatenation of "\G" and "\D". The context "\unhide{\G}" is "\G"
  with all its hidden assumptions made usable.
  % 
  The assertion "\obl{\nil ||- e}" is written simply as "e". The
  assumptions "\NEW x", "o \DEF \delta" and "\hide{o \DEF \delta}"
  \emph{bind} the identifiers "x" and "o" respectively. We write "x
  \in \G" if "x" is bound in "\G" and "x \notin \G" if "x" is not
  bound in "\G". The context "\G, h" is considered syntactically
  well-formed iff "h" does not bind an identifier already bound in
  "\G".
\end{defn}

An assertion is a statement that its goal follows from the
assumptions in its context. \tlatwo already defines such a statement
using \ASSUME \ldots \PROVE, but the context in such statements have
no hidden assumptions or definitions. (To simplify the presentation
slightly, we give the semantics of a slightly enhanced proof language
where proof steps are allowed to mention assertions instead of just
\tlatwo \ASSUME \ldots \PROVE statements.) We define an embedding of
assertions into Isabelle/\tlaplus propositions, which we take as the
ultimate primitives of the \tlatwo meta-logic.

\begin{defn} \label{defn:isabelle-embedding}
  %
  The Isabelle/\tlaplus \emph{embedding} "\isa{-}" of assertions,
  contexts and definables is as follows:
  %
  \begin{align*}
    % \begin{aligned}
    \isa{"\G ||- e"} &\ =\  "\isa{\unhide{\G}}\, e" \\[1ex]
    % \end{aligned}
    % \qquad \qquad \qquad
    % \begin{aligned}
    \isa{\nil} &\ =\  \\
    \isa{"\G, \NEW x"} &\ =\  \isa{\G}\,\And x. \\
    \isa{"\G, o \DEF \delta"} &\ =\  \isa{\G}\,\And o.\,\bigl(o \equiv \isa{\delta}\bigr)\ "==>" \\
    \isa{"\G, \phi"} &\ =\  \isa{\G}\,\bigl(\isa{\phi}\bigr)\ "==>" \\[1ex]
    % \end{aligned}
    \isa{"\LAMBDA\ \vec x : e"} &\ =\ "\lambda \vec x.\ e" % \\[1ex]
  \end{align*}
\end{defn}

\noindent
For example, "\isa{\NEW P, \hide{\obl{\NEW x ||- P(x)}} ||- \forall x
  : P(x)} = \And P.\ \left(\And x.\ P(x)\right) ==> \forall x : P(x)".
Note that usable and hidden assumptions are treated identically for
%% SM: changed following line
%the truth of an assertion.
the translation of an assertion to Isabelle/\tlaplus.

The embedding of ordinary \tlatwo expressions is the identity because
Isabelle/\tlaplus contains \tlatwo expressions as part of its object
syntax. Thus, we do not have to trust the embedding of ordinary
\tlatwo expressions, just that of the assertion language.  In
practice, some aspects of \tlatwo expressions, such as the
indentation-sensitive conjunction and disjunction lists, are sent by
the \PM to Isabelle using an indentation-insensitive
encoding. Moreover, as already mentioned, Isabelle/\tlaplus accepts
only constant level expressions. While Isabelle/\tlaplus can
implicitly generalize over the free identifiers in a lemma, we shall
be explicit about binding and restrict attention to closed assertions.

\begin{defn}[Provability] \label{defn:true} \mbox{}
  %
  The assertion "\obl{\G ||- e}" is said to be \emph{provable} iff it is
  closed and "\isa{\G ||- e}" is a well-typed proposition that has a
  proof in Isabelle/\tlaplus.
\end{defn}

We state, omitting the trivial proofs, a number of useful facts about
assertions (which are all theorems in Isabelle/\tlaplus). The last one
(fact~\ref{thm:classic}) is true because \tlaplus is based on classical logic.

\begin{fac}[Definition] \label{thm:definition}
  % 
  If "\obl{\G, \NEW o, \D ||- e}" is provable, then "\obl{\G, o \DEF
    \delta, \D ||- e}" is true.
\end{fac}

\begin{fac}[Weakening] \label{thm:weaken}
  %
  If "\obl{\G, \D ||- e}" is provable, then "\obl{\G, h, \D ||- e}" is
  provable.
\end{fac}

\begin{fac}[Expansion] \label{thm:expand}
  %
  If "\obl{\G, o \DEF \delta, \D ||- e}" is provable, then "\obl{\G, o
    \DEF \delta, \D[o := \delta] ||- e[o := \delta]}" is provable.
\end{fac}

\begin{fac}[Strengthening] \label{thm:delete}
  %
  If "\obl{\G, \NEW o, \D ||- e}" or "\obl{\G, o \DEF \delta, \D ||-
    e}" is provable and "o"
  is not free in "\obl{\D ||- e}", then "\obl{\G, \D ||- e}" is provable.
\end{fac}

\begin{fac}[Cut] \label{thm:cut}
  %
  If "\obl{\G, \D ||- e}" is provable and "\obl{\G, (\D ||- e), \W ||- f}" is true,
  then "\obl{\G, \W ||- f}" is provable.
\end{fac}

\begin{fac} \label{thm:classic}
  %
  If "\obl{\G, \lnot e, \D ||- e}" is provable, then "\obl{\G, \D ||- e}"
  is provable.
\end{fac}

The \USE/\HIDE \DEFS steps change the visibility of definitions in a
context (Defn.~\ref{defn:use/hide} below). Note that changing the
visibility of a definition does not affect the provability of an assertion
because the Isabelle embedding (Defn.~\ref{defn:isabelle-embedding})
makes all hidden definitions usable.

\pagebreak[2]

\begin{defn} \label{defn:use/hide}
  %
  If "\G" is a context and "\P" a set of operator names, then:
  \begin{ecom}
  \item \emph{"\G" with "\P" made usable}, written "\G \USING \P", is
    constructed from "\G" by replacing all assumptions of the form
    "\hide{o \DEF \delta}" in "\G" with "o \DEF \delta" for every "o
    \in \P".
  \item \emph{"\G" with "\P" made hidden}, written "\G \HIDING \P", is
    constructed from "\G" by replacing all assumptions of the form "o
    \DEF \delta" in "\G" with "\hide{o \DEF \delta}" for every "o \in
    \P".
  \end{ecom}
\end{defn}

\def\refl#1{\bigl\|\,#1\,\bigr\|}

A sequence of binders "\vec \beta" in the \tlatwo expressions "\forall
\vec \beta : e" or "\exists \vec \beta : e" can be reflected as
assumptions.

\begin{defn}[Binding Reflection] \label{defn:binding-reflection}
  %
  If "\vec \beta" is a list of binders with each element of the form
  "x" or "x \in e", then the \emph{reflection} of "\vec \beta" as
  assumptions, written "\refl{\vec \beta}", is given inductively as
  follows.
  \begin{align*}
    \refl{\nil} &= \nil &
    \refl{\vec \beta, x} &= \refl{\vec \beta}, \NEW x &
    \refl{\vec \beta, x \in e} &= \refl{\vec \beta}, \NEW x, x \in e
  \end{align*}
\end{defn}

\subsection{Interpreting Proofs}
\label{apx:proof-transformation}

Let us recall some definitions from section~\ref{sec:obligations}.

\begin{defn}[Claims and Transformations] \label{defn:check/trans-def}
  %
  A \emph{claim} is a judgement of the form "\pi : \obl{\G ||- e}"
  where "\pi" is a \tlatwo proof. A \emph{transformation} is a
  judgement of the form "\sigma.\,\tau : \obl{\G ||- e} --> \obl{\D
    ||- f}" where "\sigma" is a begin-step token and "\tau" a proof
  step.
\end{defn}

The \PM generates proof obligations from claims using two mutually
recursive procedures, \textit{checking} and
\textit{transformation}. Each procedure is specified below using the
formalism of a \textit{primitive derivation}.

\begin{defn} \label{defn:primitive}
  %
  A \emph{primitive derivation} is a derivation constructed using
  inferences
  % note to concerned citizens: I don't mean "inference rule" here. A
  % derivation is made up of a tree of inferences, with each inference
  % following a schematic inference rule.
  of the form
  \begin{gather*}
    \Ic{E}{\DD_1 & \dotsb & \DD_n} \tag*{"(n \ge 0)"}
  \end{gather*}
  where "E" is either a claim or a transformation, and "\DD_1, ...,
  \DD_n" are primitive derivations or assertions. An assertion at the
  leaf of a primitive derivation is called a \emph{proof obligation},
  and we say that a primitive derivation \emph{generates} the set of
  proof obligations at its leaves.
\end{defn}

\begin{defn}[Checking and Transformation Procedures] \label{defn:check/trans-proc} \mbox{}
  \begin{ecom}
  \item The \tlatwo proof "\pi" \emph{checks} the assertion "\obl{\G
      ||- e}" if there is a primitive derivation concluding "\pi :
    \obl{\G ||- e}" using the inference rules below and transformation
    rules.
    %
    \begin{gather*}
      % \I[\OBVIOUS]{\OBVIOUS : \obl{"\G ||- e"}}{\OBVIOUS : \obl{"\G ||- e"}}
      % \SP
      % \I[\OMITTED]{\OMITTED : \obl{"\G ||- e"}}{\OMITTED : \obl{"\G ||- e"}}
      % \\[1ex]
      \I[\BY]{"\BY\ \Phi\ \DEFS\ \P : (\G
        ||- e)"}
        {"\s{0}.\ \USE\ \Phi\ \DEFS\ \P" : "(\G ||- e) --> (\D ||- f)"
         &
         "\OBVIOUS : (\D ||- f)"}
      \\[1ex]
      \I[\QED]{"\sigma.\ \QED\ \PROOF\ \pi : (\G ||- e)"}
        {"\pi : (\G ||- e)"}
      \SP
      \I[non-\QED]{"\sigma.\,\tau\ \ \Pi : (\G ||- e)"}
        {"\sigma.\,\tau : (\G ||- e) --> (\D ||- f)"
         &
         "\Pi : (\D ||- f)"}
    \end{gather*}  
  \item 
    The \tlatwo proof step "\sigma.\,\tau" \emph{transforms} the
    assertion "\obl{\G ||- e}" into "\obl{\D ||- f}" if there is a
    primitive derivation concluding "\sigma.\,\tau : \obl{\G ||- e} -->
    \obl{\D ||- f}" using the inference rules below and checking rules.
    %
    \begin{gather*}
      \I[\USE\ \DEFS]{"\sigma.\ \USE\ \Phi\ \DEFS\ \P : \obl{\G ||- e} --> \obl{\D ||- f}"}
        {"\sigma.\ \USE\ \Phi : \obl{\G \USING \P ||- e} --> \obl{\D ||- f}"}
      \\[1ex]
      \I[\HIDE\ \DEFS]{"\sigma.\ \HIDE\ \Phi\ \DEFS\ \P : \obl{\G ||- e} --> \obl{\D \HIDING \P ||- f}"}
         {"\sigma.\ \HIDE\ \Phi : \obl{\G ||- e} --> \obl{\D ||- f}"}
      \\
      \I[\DEFINE ("o \notin \G")]{"\sigma.\ \DEFINE\ o \DEF \delta : \obl{\G ||- e} --> \obl{\G, \hide{o \DEF \delta} ||- e}"}
      \\
      \I["\USE_0"]{"\sigma.\ \USE\ \nil : \obl{\G ||- e} --> \obl{\G ||- e}"}
      \SP
      \I["\HIDE_0"]{"\sigma.\ \HIDE\ \nil : \obl{\G ||- e} --> \obl{\G ||- e}"}
      \\[1ex]
      \I["\USE_1"]{"\sigma.\ \USE\ \Phi, \obl{\G_0 ||- e_0} : \obl{\G ||- e} --> \obl{\D, \obl{\G_0 ||- e_0} ||- f}"}
        {"\sigma.\ \USE\ \Phi : \obl{\G ||- e} --> \obl{\D ||- f}"
         &
         "\OBVIOUS : \obl{\unhide{\D}, \G_0 ||- e_0}"
        }
      \\[1ex]
      \I["\HIDE_1"]{"\sigma.\ \HIDE\ \Phi, \phi : \obl{\G_0, \phi, \G_1 ||- e} --> \obl{\D ||- f}"}
        {"\sigma.\ \HIDE\ \Phi : \obl{\G_0, hide{\phi}, \G_1 ||- e} --> \obl{\D ||- f}"}
      \\
      \I["\TAKE_0"]{"\sigma.\ \TAKE\ \nil : \obl{\G ||- e} --> \obl{\G ||- e}"}
      \LSP
      \I["\WITNESS_0"]{"\sigma.\ \WITNESS\ \nil : \obl{\G ||- e} --> \obl{\G ||- e}"}
      \\[1ex]
      \I["\TAKE_1"]{"\sigma.\ \TAKE\ u, \vec\beta : \obl{\G ||- \forall x : e} --> \obl{\D ||- f}"}
        {"\sigma.\ \TAKE\ \vec\beta : \obl{\G, \NEW u ||- e [x := u]} --> \obl{\D ||- f}"}
      \\[1ex]
      \I["\TAKE_2"]{"\sigma.\ \TAKE\ u \in T, \vec\beta : \obl{\G ||- \forall x \in S : e} --> \obl{\D ||- f}"}
        {"\OBVIOUS : \obl{\G ||- S \subseteq T}"
         &
         "\sigma.\ \TAKE\ \vec\beta : \obl{\G, \NEW u, u \in T ||- e [x := u]} --> \obl{\D ||- f}"}
      \\[1ex]
      \I["\WITNESS_1"]{"\sigma.\ \WITNESS\ w, \W : \obl{\G ||- \exists x : e} --> \obl{\D ||- f}"}
        {"\sigma.\ \WITNESS\ \W : \obl{\G ||- e [x := w]} --> \obl{\D ||- f}"}
      \\[1ex]
      \I["\WITNESS_2"]{"\sigma.\ \WITNESS\ w \in T, \W : \obl{\G ||- \exists x \in S : e} --> \obl{\D ||- f}"}
        {\begin{array}[b]{c}
            "\OBVIOUS : \obl{\G ||- T \subseteq S}" \\
            "\OBVIOUS : \obl{\G ||- w \in T}"
         \end{array}
         &
         "\sigma.\ \WITNESS\ \W : \obl{\G, w \in T ||- e [x := w]} --> \obl{\D ||- f}"}
      \\[1ex]
      \I[\HAVE]{"\sigma.\ \HAVE\ g : \obl{\G ||- e => f} --> \obl{\G, g ||- f}"}
        {"\OBVIOUS : \obl{\G, e ||- g}"}
      \\[1ex]
      \I[$\ASSERT_1$]{"\s n.\ \obl{\D ||- f}\ \PROOF\ \pi : \obl{\G ||- e} --> \obl{\G, \obl{\D ||- f} ||- e}"}
        {"\pi : \obl{\G, \hide{\lnot e}, \D ||- f}"}
      \\[1ex]
      \I[$\ASSERT_2$]{"\s n l.\ \obl{\D ||- f}\ \PROOF\ \pi : \obl{\G ||- e} --> \obl{\G, \s n l \DEF \obl{\D ||- f}, \hide{\s n l} ||- e}"}
        {"\pi : \obl{\G, \s n l \DEF \obl{\D ||- f}, \hide{\lnot e}, \D ||- f}"}
      \\[1ex]
      \I[\CASE]{"\sigma.\ \CASE\ g\ \PROOF\ \pi : \obl{\G ||- e}
                       --> \obl{\D ||- f}"}
        {"\sigma.\ \obl{g ||- e}\ \PROOF\ \pi : \obl{\G ||- e} --> \obl{\D ||- f}"}
      \\[1ex]
      \I[$\SUFFICES_1$]{"\s n.\ \SUFFICES\ \obl{\D ||- f}\ \PROOF\ \pi : \obl{\G ||- e} --> \obl{\G, \hide{\lnot e}, \D ||- f}"}
        {"\pi : \obl{\G, \obl{\D ||- f} ||- e}"}
      \\[1ex]
      \I[$\SUFFICES_2$]{"\s n l.\ \SUFFICES\ \obl{\D ||- f}\ \PROOF\ \pi : \obl{\G ||- e} --> \obl{\G, \s n l \DEF \obl{\D ||- f}, \hide{\lnot e}, \D ||- f}"}
        {"\pi : \obl{\G, \s n l \DEF \obl{\D ||- f}, \hide{\s n l} ||- e}"}
      \\[1ex]
      \I[\PICK]{"\sigma.\ \PICK\ \vec\beta : p\ \PROOF\ \pi :
                  \obl{\G ||- e} --> \obl{\G, \refl{\vec \beta}, p ||- e}"}
        {"\pi : \obl{\G ||- \exists \vec \beta : p}"}
    \end{gather*}
  \end{ecom}
\end{defn}

The inference rules in the two definitions above are deterministic:
the conclusion of each rule uniquely determines the premises. However,
the rules are partial; for example, there is no rule that concludes a
transformation of the form "\sigma.\,\TAKE\ x \in S : \obl{\G ||- B \land C}
--> \obl{\D ||- f}".

\begin{defn} \label{defn:meaningful}
  % 
  A claim or a transformation is said to be \emph{meaningful} if it
  has a primitive derivation.
\end{defn}

In the rest of this appendix we limit our attention to meaningful
claims and transformations.

\subsection{Correctness}
\label{apx:correctness}

If the proof obligations generated by a meaningful claim are verifiable,
then the assertion in the claim itself ought to be true. Because
verifiability is stronger than provability and, moreover, provability
implies validity (by the assumed soundness of Isabelle/\tlaplus), it
suffices to prove that if all obligations are provable then so is the
original assertion. We will now prove this theorem by analysis of the checking and
transformation derivations. The proof of this theorem will moreover
demonstrate how to produce a proof of the ``structural theorem'' of a
claim, which is a (meta-)implication from the assertions of the proof
obligations of the claim to the assertion of the claim.

\begin{defn}[Truth of Claims and Transformation] \label{defn:proc-truth} \mbox{}
  %
  \begin{ecom}
  \item The claim "\pi : \obl{\G ||- e}" is \emph{provable} iff it is
    meaningful and the proof obligations it generates are provable.
  \item The transformation "\sigma.\,\tau : \obl{\G ||- e} --> \obl{\D
      ||- f}" is \emph{provable} iff it is meaningful and the proof
    obligations it generates are provable.
  \end{ecom}
\end{defn}

\begin{thm}[Correctness] \label{thm:correctness} \mbox{}
  %
  \begin{ecom}[\LSP (1)]
  \item If "\pi : \obl{\G ||- e}" is provable, then "\obl{\G ||- e}" is
    provable.
  \item If "\sigma.\,\tau : \obl{\G ||- e} --> \obl{\D ||- f}" is
    provable and "\obl{\D ||- f}" is provable, then "\obl{\G ||- e}" is
    provable.
  \end{ecom}
\end{thm}

\begin{proof}
  Let "\DD" be the primitive derivation for the claim in (1) and "\EE"
  be the primitive derivation for the transformation in (2). The proof
  will be by lexicographic induction on the structures of "\DD" and
  "\EE", with a provable transformation allowed to justify a
  provable claim.

  \begin{ecom}[{$\s1$}1.]
  \item If "\pi : \obl{\G ||- e}" is provable, then "\obl{\G ||-
      e}" is provable.
    %
    \begin{ecom}[{$\s2$}1.] \setlength{\itemsep}{6pt}
    \item \Case "\pi" is "\BY\ \Phi\ \DEFS\ \P", \ie,
      \begin{gather*}
        \DD = 
        \Im[\BY.]{"\BY\ \Phi\ \DEFS\ \P : (\G ||- e)"}
           {\deduce{"\s{0}.\ \USE\ \Phi\ \DEFS\ \P" : "(\G ||- e) --> (\D ||- f)"}{\EE_0}
            &
            "\OBVIOUS : (\D ||- f)"}
      \end{gather*}
      \begin{ecom}[{$\s3$}1.]
      \item "\obl{\D ||- f}" is provable
        %
        \by Def.~\ref{defn:proc-truth}.
      \item \Qed
        %
        \by \s31, i.h. (inductive hypothesis) for "\EE_0".
      % \item \Qed
      %   %
      %   \by \s31, \s32.
      \end{ecom}

    \item \Case "\pi" is "\sigma.\, \QED\ \PROOF\ \pi_0", \ie, 
      $
      \DD =
      \Im[\QED.]{"\sigma.\ \QED\ \PROOF\ \pi_0 : (\G ||- e)"}
         {\deduce{"\pi_0 : (\G ||- e)"}{\DD_0}}$
      %
      \by i.h. for "\DD_0".

    \item \Case "\pi" is "\sigma.\,\tau\ \ \Pi", \ie,
      \begin{gather*}
        \DD =
        \Im[non-\QED.]{"\sigma.\,\tau\ \ \Pi : (\G ||- e)"}
          {\deduce{"\sigma.\,\tau : (\G ||- e) --> (\D ||- f)"}{\EE_0}
           &
           \deduce{"\Pi : (\D ||- f)"}{\DD_0}
          }
      \end{gather*}

      \begin{ecom}[{$\s3$}1.]
      \item [\s31.] "\obl{\D ||- f}" is provable
        % 
        \by i.h. for "\DD_0".
      \item [\s33.] \Qed
        %
        \by \s31, i.h. for "\EE_0".
      \end{ecom}
    \item \Qed \by \s21, \s22, \s23.
    \end{ecom}

  \item If "\sigma.\,\tau : \obl{\G ||- e} --> \obl{\D ||- f}" is
    provable and "\obl{\D ||- f}" is provable, then "\obl{\G ||- e}" is
    provable.
    %
    \begin{ecom}[{$\s2$}1.] \setlength{\itemsep}{6pt}

    \item \Case "\tau" is "\USE\ \Phi\ \DEFS\ \P", \ie,
      \begin{gather*}
        \EE =
        \Im[\USE\ \DEFS.]{"\sigma.\ \USE\ \Phi\ \DEFS\ \P : \obl{\G ||- e} --> \obl{\D ||- f}"}
          {\deduce{"\sigma.\ \USE\ \Phi : \obl{\G \USING \P ||- e} --> \obl{\D ||- f}"}{\EE_0}}
      \end{gather*}

      \begin{ecom}[{$\s3$}1.]
      \item "\obl{\G \USING \P ||- e}" is provable
        %
        \by i.h. for "\EE_0".
      \item \Qed
        %
        \by \s31, Defn.~\ref{defn:use/hide}.
      \end{ecom}

    \item \Case "\tau" is "\HIDE\ \Phi\ \DEFS\ \P", \ie,
      \begin{gather*}
        \EE =
        \Im[\HIDE\ \DEFS.]{"\sigma.\ \HIDE\ \Phi\ \DEFS\ \P : \obl{\G ||- e} --> \obl{\D \HIDING \P ||- f}"}
           {\deduce{"\sigma.\ \HIDE\ \Phi : \obl{\G ||- e} --> \obl{\D ||- f}"}{\EE_0}}
      \end{gather*}

      \begin{ecom}[{$\s3$}1.]
      \item "\obl{\D ||- f}" is provable
        %
        \by provability of "\obl{\D \HIDING \P ||- f}" and Defn.~\ref{defn:use/hide}.
      \item \Qed
        %
        \by \s31, i.h. for "\EE_0".
      \end{ecom}

    \item \Case "\tau" is "\DEFINE\ o \DEF \delta" with "o \notin \G", \ie, 
      \begin{gather*}
        \EE =
        \Im[\DEFINE.]{"\sigma.\ \DEFINE\ o \DEF \delta : \obl{\G ||- e} --> \obl{\G, \hide{o \DEF \delta} ||- e}"}{}
      \end{gather*}

      \begin{ecom}[{$\s3$}1.]
      \item "o" is not free in "e"
        %
        \by "o \notin \G" and closedness of "\obl{\G ||- e}".
      \item \Qed
        %
        \by \s31, strengthening (fact~\ref{thm:delete}).
      \end{ecom}

    \item \Case "\tau" is "\USE\ \nil", \ie,
      $
      \EE = 
      \Im["\USE_0".]{"\sigma.\ \USE\ \nil : \obl{\G ||- e} --> \obl{\G ||- e}"}{}
      $
      \Trivial

    \item \Case "\tau" is "\HIDE\ \nil", \ie,
      $
      \EE = 
      \Im["\HIDE_0".]{"\sigma.\ \HIDE\ \nil : \obl{\G ||- e} --> \obl{\G ||- e}"}{}
      $
      \Trivial

    \item \Case "\tau" is "\USE\ \Phi, \phi", \ie,
      \begin{gather*}
        \EE = 
        \Im["\USE_1"]{"\sigma.\ \USE\ \Phi, \obl{\G_0 ||- e_0} : \obl{\G ||- e} --> \obl{\D_0, \obl{\G_0 ||- e_0} ||- f}"}
          {\deduce{"\sigma.\ \USE\ \Phi : \obl{\G ||- e} --> \obl{\D_0 ||- f}"}{\EE_0}
           &
           "\OBVIOUS : \obl{\unhide{\D_0}, \G_0 ||- e_0}"
          }      
      \end{gather*}

      \begin{ecom}[{$\s3$}1.]
      \item "\obl{\unhide{\D_0}, \G_0 ||- e_0}" is provable
        %
        \by Def.~\ref{defn:proc-truth}.
      \item "\obl{\D_0, \G_0 ||- e_0}" is provable
        %
        \by \s31, Def.~\ref{defn:isabelle-embedding}.
      \item "\obl{\D_0 ||- f}" is provable
        %
        \by provability of "\obl{\D_0, (\G_0 ||- e_0) ||- f}", \s32, cut (fact~\ref{thm:cut}).
      \item \Qed
        %
        \by \s33, i.h. for "\EE_0"
      \end{ecom}

    \item \Case "\tau" is "\HIDE\ \Phi, \phi", \ie,
      \begin{gather*}
        \EE =
        \Im["\HIDE_1".]{"\sigma.\ \HIDE\ \Phi, \phi : \obl{\G_0, \phi, \G_1 ||- e} --> \obl{\D ||- f}"}
           {\deduce{"\sigma.\ \HIDE\ \Phi : \obl{\G_0, hide{\phi}, \G_1 ||- e} --> \obl{\D ||- f}"}{\EE_0}}
      \end{gather*}

      \begin{ecom}[{$\s3$}1.]
      \item "\obl{\G_0, \hide{\phi}, \G_1 ||- e}" is provable
        %
        \by provability of "\obl{\D ||- f}", i.h. for "\EE_0".
      \item \Qed
        %
        \by \s31, "\isa{\G_0, \hide{\phi}, \G_1 ||- e} = \isa{\G_0, \phi, \G_1 ||- e}" (Def.~\ref{defn:isabelle-embedding}).
      \end{ecom}

    \item \Case "\tau" is "\TAKE\ \nil", \ie,
      $
      \EE = 
      \Im["\TAKE_0".]{"\sigma.\ \TAKE\ \nil : \obl{\G ||- e} --> \obl{\G ||- e}"}{}
      $
      \Trivial

    \item \Case "\tau" is "\WITNESS\ \nil", \ie,
      $
      \EE = 
      \Im["\WITNESS_0".]{"\sigma.\ \WITNESS\ \nil : \obl{\G ||- e} --> \obl{\G ||- e}"}{}
      $
      \Trivial

    \item \Case "\tau" is "\TAKE\ u, \vec \beta", \ie,
      \begin{gather*}
        \EE =
        \Im["\TAKE_1".]{"\sigma.\ \TAKE\ u, \vec\beta : \obl{\G ||- \forall x : e} --> \obl{\D ||- f}"}
           {\deduce{"\sigma.\ \TAKE\ \vec\beta : \obl{\G, \NEW u ||- e [x := u]} --> \obl{\D ||- f}"}{\EE_0}}
      \end{gather*}

      \begin{ecom}[{$\s3$}1.]
      \item "\obl{\G, \NEW u ||- e [x := u]}" is provable
        %
        \by i.h. for "\EE_0".
      \item \Qed
        %
        \by \s31{} and predicate logic.
      \end{ecom}

    \item \Case "\tau" is "\sigma.\ \TAKE\ u \in T", \ie,
      \begin{gather*}
        \EE =
        \Im["\TAKE_2".]{"\sigma.\ \TAKE\ u \in T, \vec\beta : \obl{\G ||- \forall x \in S : e} --> \obl{\D ||- f}"}
           {"\OBVIOUS : \obl{\G ||- S \subseteq T}"
            &
            \deduce{"\sigma.\ \TAKE\ \vec\beta : \obl{\G, \NEW u, u \in T ||- e [x := u]} --> \obl{\D ||- f}"}{\EE_0}}
      \end{gather*}

      \begin{ecom}[{$\s3$}1.]
      \item "\obl{\G, \NEW u, u \in T ||- e [x := u]}" is provable
        %
        \by i.h on "\EE_0".
      \item "\obl{\G, \NEW u, u \in S ||- u \in T}" is provable
        \begin{ecom}[{$\s4$}1.]
        \item "\obl{\G, \NEW u ||- S \subseteq T}" is provable
          %
          \by Def.~\ref{defn:proc-truth}, weakening
          (fact~\ref{thm:weaken}).
        \item \Qed
          %
          \by \s41, Def. of "\subseteq".
        \end{ecom}
      \item "\obl{\G, \NEW u, u \in S ||- e[x := u]}" is provable
        %
        \by \s31, \s32, cut (fact~\ref{thm:cut}).
      \item \Qed
        %
        \by \s33{} and predicate logic.
      \end{ecom}

    \item \Case "\tau" is "\WITNESS\ w, \W", \ie,
      \begin{gather*}
        \EE =
        \Im["\WITNESS_1".]{"\sigma.\ \WITNESS\ w, \W : \obl{\G ||- \exists x : e} --> \obl{\D ||- f}"}
           {\deduce{"\sigma.\ \WITNESS\ \W : \obl{\G ||- e [x := w]} --> \obl{\D ||- f}"}{\EE_0}}
      \end{gather*}

      \begin{ecom}[{$\s3$}1.]
      \item "\obl{\G ||- e [x := w]}" is provable
        %
        \by i.h. for "\EE_0".
      \item \Qed
        %
        \by \s31.
      \end{ecom}

    \item \Case "\tau" is "\WITNESS\ w \in T, \W" and:
      \begin{gather*}
        \EE =
        \Im["\WITNESS_2".]
           {"\sigma.\ \WITNESS\ w \in T, \W : \obl{\G ||- \exists x \in S : e} --> \obl{\D ||- f}"}
           {\begin{array}[b]{c}
               "\OBVIOUS : \obl{\G ||- T \subseteq S}" \\
               "\OBVIOUS : \obl{\G ||- w \in T}"
            \end{array}
            &
            \deduce{"\sigma.\ \WITNESS\ \W : \obl{\G, w \in T ||- e [x := w]} --> \obl{\D ||- f}"}{\EE_0}}
     \end{gather*}

     \begin{ecom}[{$\s3$}1.]
     \item "\obl{\G, w \in T ||- e [x := w]}" is provable
       % 
       \by i.h. for "\EE_0".
     \item "\obl{\G ||- w \in T}" is provable
       % 
       \by Def.~\ref{defn:proc-truth}.
     \item "\obl{\G ||- e [x := w]}" is provable
       % 
       \by \s31, \s32, cut (fact~\ref{thm:cut}).
     \item "\obl{\G ||- w \in S}" is provable
       \begin{ecom}[{$\s4$}1.]
       \item "\obl{\G, w \in T ||- w \in S}" is provable
         % 
         \by Def.~\ref{defn:proc-truth}, Def. of "\subseteq".
       \item \Qed
         % 
         \by \s41, \s32, cut (fact~\ref{thm:cut}).
       \end{ecom}
     \item \Qed
       % 
       \by \s33, \s34, and predicate logic.
     \end{ecom}
     
    \item "\tau" is "\HAVE\ g", \ie,
      \begin{gather*}
        \EE =
        \Im[\HAVE.]{"\sigma.\ \HAVE\ g : \obl{\G ||- e => f} --> \obl{\G, g ||- f}"}
           {"\OBVIOUS : \obl{\G, e ||- g}"}
      \end{gather*}

      \begin{ecom}[{$\s3$}1.]
      \item "\obl{\G, e, g ||- f}" is provable
        %
        \by weakening (fact~\ref{thm:weaken}).
      \item "\obl{\G, e ||- g}" is provable
        %
        \by Def.~\ref{defn:proc-truth}.
      \item "\obl{\G, e ||- f}" is provable
        %
        \by \s31, \s32, cut (fact~\ref{thm:cut}).
      \item "\obl{\G ||- e => f}" is provable
        %
        \by \s33.
      \end{ecom}

    \item "\sigma.\,\tau" is "\s n.\ \obl{\W ||- g}\ \PROOF\ \pi", \ie, 
      \begin{gather*}
        \EE =
        \Im[$\ASSERT_1$.]{"\s n.\ \obl{\W ||- g}\ \PROOF\ \pi : \obl{\G ||- e} --> \obl{\G, \obl{\W ||- g} ||- e}"}
           {\deduce{"\pi : \obl{\G, \hide{\lnot e}, \W ||- g}"}{\DD_0}}
      \end{gather*}

      \begin{ecom}[{$\s3$}1.]
      \item "\obl{\G, \hide{\lnot e}, \obl{\W ||- g} ||- e}" is provable
        %
        \by weakening (fact~\ref{thm:weaken}).
      \item "\obl{\G, \hide{\lnot e}, \W ||- g}" is provable
        %
        \by i.h. for "\DD_0".
      \item "\obl{\G, \hide{\lnot e} ||- e}" is provable
        %
        \by \s31, \s32, cut (fact~\ref{thm:cut}).
      \item \Qed
        %
        \by \s33, fact~\ref{thm:classic}.
      \end{ecom}

    \item \Case "\sigma.\,\tau" is "\s n l.\ \obl{\W ||- g}\ \PROOF\ \pi", \ie,
      \begin{gather*}
        \EE =
        \Im[$\ASSERT_2$.]{"\s n l.\ \obl{\W ||- g}\ \PROOF\ \pi : \obl{\G ||- e} --> \obl{\G, \s n l \DEF \obl{\W ||- g}, \hide{\s n l} ||- e}"}
           {\deduce{"\pi : \obl{\G, \s n l \DEF \obl{\W ||- g}, \hide{\lnot e}, \W ||- g}"}{\DD_0}}
      \end{gather*}

      \begin{ecom}[{$\s3$}1.]
      \item "\obl{\G, \s n l \DEF \obl{\W ||- g}, \hide{\lnot e}, \hide{\s n l} ||- e}" is provable
        \\\mbox{}
        \by provability of "\obl{\G, \s n l \DEF \obl{\W ||- g}, \hide{\s n l} ||- e}", weakening (fact~\ref{thm:weaken}).
      \item "\obl{\G, \s n l \DEF \obl{\W ||- g}, \hide{\lnot e}, \hide{\obl{\W ||- g}} ||- e}" is provable
        %
        \by \s31, expansion (fact~\ref{thm:expand}).
      \item "\obl{\G, \s n l \DEF \obl{\W ||- g}, \hide{\lnot e}, \W ||- g}" is provable
        % 
        \by i.h. for "\DD_0", weakening.
      \item "\obl{\G, \s n l \DEF \obl{\W ||- g}, \hide{\lnot e} ||- e}" is provable
        %
        \by \s32, \s33, cut (fact~\ref{thm:cut}).
      \item "\obl{\G, \hide{\lnot e} ||- e}" is provable
        %
        \by \s34, strengthening (fact~\ref{thm:delete}).
      \item \Qed
        %
        \by \s35, fact~\ref{thm:classic}.
      \end{ecom}
      
    \item "\tau" is "\CASE\ g\ \PROOF\ \pi", \ie,
      \begin{gather*}
        \EE =
        \Im[\CASE.]{"\sigma.\ \CASE\ g\ \PROOF\ \pi : \obl{\G ||- e} --> \obl{\D ||- f}"}
           {\deduce{"\sigma.\ \obl{g ||- e}\ \PROOF\ \pi : \obl{\G ||- e} --> \obl{\D ||- f}"}{\EE_0}}
      \end{gather*}

      \textit{By} i.h. for "\EE_0".

    \item "\tau" is "\s n.\ \SUFFICES\ \obl{\W ||- g}\ \PROOF\ \pi", \ie,
      \begin{gather*}
        \EE =
        \Im[$\SUFFICES_1$.]{"\s n.\ \SUFFICES\ \obl{\D ||- f}\ \PROOF\ \pi : \obl{\G ||- e} --> \obl{\G, \hide{\lnot e}, \W ||- g}"}
           {\deduce{"\pi : \obl{\G, \obl{\W ||- g} ||- e}"}{\DD_0}}
      \end{gather*}

    \begin{ecom}[{$\s3$}1.]
      \item "\obl{\G, \hide{\lnot e}, \obl{\W ||- g} ||- e}" is provable
        %
        \by i.h. for "\DD_0", weakening (fact~\ref{thm:weaken}).
      \item "\obl{\G, \hide{\lnot e} ||- e}" is provable
        %
        \by provability of "\obl{\G, \hide{\lnot e}, \W ||- g}", \s31, cut (fact~\ref{thm:cut}).
      \item \Qed
        %
        \by \s32, fact~\ref{thm:classic}.
      \end{ecom}

    \item "\sigma.\,\tau" is "\s n l.\ \SUFFICES\ \obl{\W ||- g}\ \PROOF\ \pi", \ie,
      \begin{gather*}
        \EE =
        \Im[$\SUFFICES_2$.]{"\s n l.\ \SUFFICES\ \obl{\W ||- g}\ \PROOF\ \pi : \obl{\G ||- e} --> \obl{\G, \s n l \DEF \obl{\W ||- g}, \hide{\lnot e}, \W ||- g}"}
           {\deduce{"\pi : \obl{\G, \s n l \DEF \obl{\W ||- g}, \hide{\s n l} ||- e}"}{\DD_0}}
      \end{gather*}

      \begin{ecom}[{$\s3$}1.]
      \item "\obl{\G, \s n l \DEF \obl{\W ||- g}, \hide{\lnot e}, \hide{\s n l} ||- e}" is provable
        %
        \by i.h. for "\DD_0", weakening (fact~\ref{thm:weaken}).
      \item "\obl{\G, \s n l \DEF \obl{\W ||- g}, \hide{\lnot e}, \hide{\obl{\W ||- g}} ||- e}" is provable
        %
        \by \s31, expansion (fact~\ref{thm:expand}).
      \item "\obl{\G, \s n l \DEF \obl{\W ||- g}, \hide{\lnot e} ||- e}" is provable
        \\\mbox{}
        \by \s32, provability of "\obl{\G, \s n l \DEF \obl{\W ||- g}, \hide{\lnot e}, \W ||- g}", cut (fact~\ref{thm:cut}).
      \item "\obl{\G, \hide{\lnot e} ||- e}" is provable
        %
        \by \s33, strengthening (fact~\ref{thm:delete}).
      \item \Qed
        %
        \by \s34, fact~\ref{thm:classic}.
      \end{ecom}

    \item \Case "\tau" is "\PICK\ \vec\beta : p\ \PROOF\ \pi", \ie,
      \begin{gather*}
        \EE =
        \Im[\PICK.]{"\sigma.\ \PICK\ \vec\beta : p\ \PROOF\ \pi :
                        \obl{\G ||- e} --> \obl{\G, \refl{\vec \beta}, p ||- e}"}
           {\deduce{"\pi : \obl{\G ||- \exists \vec \beta : p}"}{\DD_0}}
      \end{gather*}

      \begin{ecom}[{$\s3$}1.]
      \item "\obl{\G, \exists \vec \beta : p ||- e}" is provable
        %
        \by provability of "\obl{\G, \refl{\vec \beta}, p ||- e}", predicate logic.
      \item "\obl{\G ||- \exists \vec \beta : p}" is provable
        %
        \by i.h. for "\DD_0".
      \item \Qed
        %
        \by \s31, \s32, cut (fact~\ref{thm:cut}).
      \end{ecom}

    \item \Qed
      %
      \by \s21, \dots, \s220
    \end{ecom}
  \item \Qed
    %
    \by \s11, \s12.
  \end{ecom}
\end{proof}

\subsection{Verification}
\label{apx:constraints}

The correctness theorem (\ref{thm:correctness}) establishes an
implication from the proof obligations generated by a meaningful claim
to its assertion. It holds regardless of the verifiability
of any individual obligation. While changing the visibility of
assumptions in an assertion does not change its provability, it can change
its verifiability if important assumptions are hidden because hidden
assumptions are not passed to the back-end provers.  
% As already
% mentioned in Section~\ref{sec:obligations}, the \PM removes these
% hidden assumptions before sending a proof obligation to a back-end
% prover.  
Therefore, in order to establish the structural soundness
theorem (\ref{thm:meaning}), we must prove a property about the result
of this removal.

\begin{defn}[Filtration] \label{defn:filter} 
  %
  The \emph{filtered} form of any assertion "\phi", written
  "\filter\phi", is obtained by deleting all assumptions of the form
  "\hide{\phi_0}" and replacing all assumptions of the form "\hide{o
    \DEF \delta}" with "\NEW o" anywhere inside "\phi".
\end{defn}

For example, if "\phi" is "\obl{\NEW x, \hide{y \DEF x} ||- x = y}",
then "\filter\phi = \obl{\NEW x, \NEW y ||- x = y}". Clearly,
filtration makes an assertion implicationally weaker, for "\phi" is
provable while "\filter\phi" is not.

% \begin{defn}[Filtered Provability] \label{defn:verifiable}
%   %
%   The assertion "\phi" is said to be \emph{filtered-provable} if
%   "\filter{\phi}" is provable.
% \end{defn}

\begin{lem}[Filtration Lemma] \label{thm:verification}
  %
  If "\filter{\phi}" is provable, then "\phi" is provable.
\end{lem}

\begin{proof}[Proof Sketch]
  By induction on the structure of the assertion "\phi", with each
  case a straightforward consequence of facts~\ref{thm:definition}
  and~\ref{thm:weaken}.
\end{proof}

We now prove the structural soundness theorem (\ref{thm:meaning}).

\setcounter{thm}{0}

\begin{thm}
  % 
  If "\pi:\phi" is a meaningful claim and the proof obligations it
  generates are all verifiable, then "\phi" is true.
\end{thm}

\begin{proof}\mbox {}
  \begin{ecom}[{$\s1$}1.]
  \item For every proof obligation "\phi_0" generated by "\pi : \phi",
    it must be that "\phi_0" is provable.
    \begin{ecom}[{$\s2$}1.]
    \item \textit{Take} "\phi_0" as a proof obligation generated by
      "\pi : \phi".
    \item "\filter{\phi_0}" is provable \by verifiability of "\phi_0".
    \item \Qed \by \s22, filtration lemma~\ref{thm:verification}.
    \end{ecom}
  \item "\pi : \phi" is provable \by correctness theorem~\ref{thm:correctness}.
  \item \Qed \by soundness of Isabelle/\tlaplus.
  \end{ecom}
\end{proof}

% Commented this out as it's already explained in body text. -- KC
%
% In order to conclude from a meaningful claim "\pi : \obl{\G ||- e}"
% that "\obl{\G ||- e}" is indeed true, the \PM performs the following
% steps:
% %
% \begin{ecom}
% \item Generate the proof obligations for the claim.
% \item In every \OBVIOUS obligation, filter its assertion
%   (Defn.~\ref{defn:filter}) and ask a back-end prover to prove it,
%   certifying the proof in Isabelle/\tlaplus if successful. This also
%   results in a proof of the unfiltered assertion by
%   Lem.~\ref{thm:verification} (which Isabelle can automatically prove).
% \item Prove (in Isabelle/\tlaplus) the structural obligation, which is
%   an Isabelle implication ("==>") from the embedding of the assertions
%   in the proof obligations of the claim to the Isabelle/\tlaplus
%   proposition "\isa{\G ||- e}". This Isabelle/\tlaplus proof is
%   constructed by rendering the proof of the corresponding cases of the
%   correctness theorem 
% % LL:  We don't have to keep repeating this.
% % (\ref{thm:correctness}) 
% in Isar.
% \item If there are no \OMITTED obligations and all \OBVIOUS
%   obligations were certified in step 2, compose them with the
%   structural obligation. This yields a certified proof of "\isa{\G ||-
%     e}".
% \end{ecom}
% %
% The Isabelle kernel and the translation "\isa{-}" are trusted, so the
% \PM concludes that "\obl{\G ||- e}" is true.

\clearpage

\section{A \tlatwo Proof of Cantor's Theorem}
\label{apx:cantor}

The following is the full \tlatwo proof of Cantor's theorem.
%
\begin{quote} \small
  \begin{tabbing}
    \THEOREM\ "\forall S : \forall f \in [S -> \SUBSET\ S] : \exists A \in \SUBSET\ S : \forall x \in S : f[x] \neq A" \\
    \s11.\ \= \ASSUME \= "\NEW\ S", \\
           \>         \> "\NEW\ f \in [S -> \SUBSET\ S]" \\
           \> \PROVE "\exists A \in \SUBSET\ S : \forall x \in S : f[x] \neq A" \\
           \> \s21.\ \= \DEFINE "T \DEF \{z \in S : z \notin f[z]\}" \\
           \> \s22.  \> "\forall x \in S : f[x] \neq T" \\
           \>        \> \s31.\ \= \ASSUME "\NEW\ x \in S" \PROVE "f[x] \neq T" \\
           \>        \>        \> \s41.\ \= \CASE "x \in T" \OBVIOUS \\
           \>        \>        \> \s42.\ \> \CASE "x \notin T" \OBVIOUS \\
           \>        \>        \> \s43.\ \> \QED \BY \s41, \s42 \\
           \>        \> \s32.  \> \QED\ \BY\ \s31 \\
           \> \s23.  \> \QED\ \BY\ \s22 \\
    \s12.  \> \QED\ \BY\ \s11
  \end{tabbing}
\end{quote}
%
As an example, the obligation generated (see Appendix~\ref{apx:correctness}) for the proof of \s41 is:
%
\begin{quote} \small
  \begin{tabbing}
    "\OBVIOUS : \smash{\Bigl(}"
    \= "\s11 \DEF \obl{\NEW S, \NEW f, f \in [S -> \SUBSET\ S] ||- \exists A \in \SUBSET\ S : \forall x \in S : f[x] \neq A}", \\
    \> "\NEW S", \\
    \> "\NEW f", "f \in [S -> \SUBSET\ S]", \\
    \> "T \DEF \{z \in S : z \notin f[z]\}", \\
    \> "\hide{\lnot \left(\exists A \in \SUBSET\ S : \forall x \in S : f[x] \neq A\right)}", \\
    \> "\s22 \DEF \forall x \in S : f[x] \neq T", \\
    \> "\hide{\lnot \left( \forall x \in S : f[x] \neq T \right)}", \\
    \> "\s31 \DEF \obl{\NEW x, x \in S ||- f[x] \neq T}", \\
    \> "\NEW x", "x \in S", \\
    \> "\hide{\lnot \left( f[x] \neq T \right)}", \\
    \> "\s41 \DEF \obl{x \in T ||- f[x] \neq T}", \\
    \> "x \in T" \\
    \> "||- f[x] \neq T" "\smash{\Bigr)}".
  \end{tabbing}
\end{quote}
%
Filtering its assertion (see Def.~\ref{defn:filter}) and expanding all definitions gives:
%
\begin{quote} \small
  \begin{tabbing}
    "\smash{\Bigl(}"
    \= "\NEW S", \\
    \> "\NEW f", "f \in [S -> \SUBSET\ S]", \\
    \> "\NEW x", "x \in S", \\
    \> "x \in \{z \in S : z \notin f[z]\} ||- f[x] \neq \{z \in S : z \notin f[z]\}" "\smash{\Bigr)}".
  \end{tabbing}
\end{quote}
%
In Isabelle/\tlaplus, this is the following lemma:
\begin{quote} \small
  \begin{tabbing}
    \texttt{lemma} \= "!! S." \\
                   \> "!! f.\ " \= " f \in [S -> \SUBSET\ S] ==>" \\
                   \>                 \> "\Bigl(!! x.\ \bigl\llbracket" \= "x \in S ;" \\
                   \> \> \> "x \in \{z \in S : z \notin f[z]\} \ \bigr\rrbracket
                         ==> f[x] \neq \{z \in S : z \notin f[z]\} \Bigr)".
  \end{tabbing}
\end{quote}



%%%% PLEASE DO NOT EDIT BELOW THIS LINE
\ifx\master\undefined
{\let\master\relax%%% -*- mode: LaTeX; TeX-master: "main.tex"; -*-

\ifx\master\undefined
\documentclass[a4paper]{easychair}
\usepackage{submission}
\begin{document}
{\let\master\relax %%% -*- mode: LaTeX; TeX-master: "main.tex"; -*-

\ifx\master\undefined
\documentclass[a4paper]{easychair}
\usepackage{submission}
\begin{document}
\fi
%%%% PLEASE DO NOT EDIT ABOVE THIS LINE

\title{A \tlaplus Proof System}

\titlerunning{A \tlaplus Proof System}

% \volumeinfo
% 	{P. Rudnicki, G. Sutcliffe} % editors
% 	{2}                         % number of editors
% 	{KEAPPA 2008}               % event
% 	{1}                         % volume
% 	{1}                         % issue
% 	{1}                         % starting page number


%  Alphabetically by surname
\author{
  Kaustuv Chaudhuri \\
  INRIA \\
  \and
  Damien Doligez \\
  INRIA \\
  \and
  Leslie Lamport \\
  Microsoft Research \\
  \and
  Stephan Merz \\
  INRIA \& Loria
}

\authorrunning{Chaudhuri, Doligez, Lamport, and Merz}

\maketitle

%%%% PLEASE DO NOT EDIT BELOW THIS LINE
\ifx\master\undefined
{\let\master\relax \input{rearmatter}}
\end{document}
\fi

% LocalWords:  tex Rudnicki Sutcliffe KEAPPA Kaustuv Chaudhuri INRIA Doligez
% LocalWords:  Merz Loria
}
\fi
%%%% PLEASE DO NOT EDIT ABOVE THIS LINE

\bibliographystyle{plain}
\bibliography{submission}

%%%% PLEASE DO NOT EDIT BELOW THIS LINE
\ifx\master\undefined
\end{document}
\fi

% LocalWords:  tex Paxos
}
\end{document}
\fi
                % checking/transformation details
                                % proof of correctness

\ifdraft\else
\clearpage
\tableofcontents
\fi
\end{document}

