%%% -*- mode: LaTeX; TeX-master: "main.tex"; -*-

\ifx\master\undefined
\documentclass[a4paper]{easychair}
\usepackage{submission}
\begin{document}
{\let\master\relax%%% -*- mode: LaTeX; TeX-master: "main.tex"; -*-

\ifx\master\undefined
\documentclass[a4paper]{easychair}
\usepackage{submission}
\begin{document}
\fi
%%%% PLEASE DO NOT EDIT ABOVE THIS LINE

\title{A \tlaplus Proof System}

\titlerunning{A \tlaplus Proof System}

% \volumeinfo
% 	{P. Rudnicki, G. Sutcliffe} % editors
% 	{2}                         % number of editors
% 	{KEAPPA 2008}               % event
% 	{1}                         % volume
% 	{1}                         % issue
% 	{1}                         % starting page number


%  Alphabetically by surname
\author{
  Kaustuv Chaudhuri \\
  INRIA \\
  \and
  Damien Doligez \\
  INRIA \\
  \and
  Leslie Lamport \\
  Microsoft Research \\
  \and
  Stephan Merz \\
  INRIA \& Loria
}

\authorrunning{Chaudhuri, Doligez, Lamport, and Merz}

\maketitle

%%%% PLEASE DO NOT EDIT BELOW THIS LINE
\ifx\master\undefined
{\let\master\relax %%% -*- mode: LaTeX; TeX-master: "main.tex"; -*-

\ifx\master\undefined
\documentclass[a4paper]{easychair}
\usepackage{submission}
\begin{document}
{\let\master\relax %%% -*- mode: LaTeX; TeX-master: "main.tex"; -*-

\ifx\master\undefined
\documentclass[a4paper]{easychair}
\usepackage{submission}
\begin{document}
\fi
%%%% PLEASE DO NOT EDIT ABOVE THIS LINE

\title{A \tlaplus Proof System}

\titlerunning{A \tlaplus Proof System}

% \volumeinfo
% 	{P. Rudnicki, G. Sutcliffe} % editors
% 	{2}                         % number of editors
% 	{KEAPPA 2008}               % event
% 	{1}                         % volume
% 	{1}                         % issue
% 	{1}                         % starting page number


%  Alphabetically by surname
\author{
  Kaustuv Chaudhuri \\
  INRIA \\
  \and
  Damien Doligez \\
  INRIA \\
  \and
  Leslie Lamport \\
  Microsoft Research \\
  \and
  Stephan Merz \\
  INRIA \& Loria
}

\authorrunning{Chaudhuri, Doligez, Lamport, and Merz}

\maketitle

%%%% PLEASE DO NOT EDIT BELOW THIS LINE
\ifx\master\undefined
{\let\master\relax \input{rearmatter}}
\end{document}
\fi

% LocalWords:  tex Rudnicki Sutcliffe KEAPPA Kaustuv Chaudhuri INRIA Doligez
% LocalWords:  Merz Loria
}
\fi
%%%% PLEASE DO NOT EDIT ABOVE THIS LINE

\bibliographystyle{plain}
\bibliography{submission}

%%%% PLEASE DO NOT EDIT BELOW THIS LINE
\ifx\master\undefined
\end{document}
\fi

% LocalWords:  tex Paxos
}
\end{document}
\fi

% LocalWords:  tex Rudnicki Sutcliffe KEAPPA Kaustuv Chaudhuri INRIA Doligez
% LocalWords:  Merz Loria
}
\fi
%%%% PLEASE DO NOT EDIT ABOVE THIS LINE

\clearpage
\appendix

\section{Context Management}
\label{sec:appendix.context}

\llnote{The syntax of the informal proofs should mirror the syntax
of \tlatwo.  Writing something like ``Cases of'' is both confusing and 
automatically raises the question of why we designed a proof language
and then didn't use it.}

In this section we shall provide the details of all context management
operations performed by the \PM. As mentioned in
section~\ref{sec:backend.pm}, hidden facts and definitions must be
filtered out of obligations, replacing definitions with declarations.
% 
% LL: We've already said that contexts are closed obligations, so
% we don't have to mention contexts here.
% 
For any obligation "\obl{\G ||- e}" we define
the \emph{filtered} obligation "\filter{\G ||- e}" to be the one defined
in the obvious way by eliminating hidden facts and replacing definitions
with declarations.

% LL: I removed the following definition because I think it's too trivial
% to be worth formalizing.

% \begin{defn} \label{defn:filter}
%   %
%   The \emph{filtration} operation "\filter{-}" on obligations and
%   contexts is as follows.
%   % 
%   \begin{gather*}
%     "\filter{\G ||- e}" = "\obl{\filter{\G} ||- e}"
%     \SP
%     \begin{aligned}[t]
%       \filter{"\G, \NEW x"} &= \filter{\G}, "\NEW x" \\
%       \filter{\G, o \DEF \hat e} &= \filter{\G}, o \DEF \filter{\hat e} \\
%       \filter{\G, \phi} &= \filter{\G}, \filter{\phi}
%     \end{aligned} \SP
%     \begin{aligned}[t]
%       \filter{\G, \hide{o \DEF \hat e}} &= \filter{\G}, "\NEW\ o" \\
%       (\G, "hide{\phi}")_f &= (\G)_f \\
%       \filter{\cdot} &= \cdot
%     \end{aligned}
%   \end{gather*}
%   %
%   An obligation is said to be \emph{filtered} if it contains no hidden
%   facts or definitions.
% \end{defn}

We begin with three obvious facts about obligations.  
%
% LL: I eliminated the proof sketches of the following three lemmas by
% adding the following sentence
%
They all follow easily by induction on the structure of a correct
proof.

\begin{lem}[Operator Substitution] \label{thm:op-subst}
  %
  If "\obl{\G, \NEW o, \D ||- e}" is true and the free identifiers of
  "\hat e" are all bound in "\G", then the following are true:
  \begin{ecom}
  \item "\obl{\G, \D[o := \hat e] ||- e[o := \hat e]}"; and
  \item "\obl{\G, o \DEF \hat e, \D ||- e}".
  \end{ecom}
\end{lem}

% \begin{proof}[Proof Sketch] 
%   By induction on the structure of a correct proof of "\obl{\G, \NEW
%     o, \D ||- e}" whose leaf obligations are instances of \tlatwo
%   axioms..
% \end{proof}

\begin{lem}[Strengthening] \label{thm:delete}
  %
  If "\obl{\G, o \DEF \hat e, \D ||- f}" or "\obl{\G, \hide{o \DEF
      \hat e}, \D ||- f}" or "\obl{\G, \NEW o, \D ||- f}" is true and
  "o" is not free in "\obl{\D ||- e}", then "\obl{\G, \D ||- e}" is
  true.
\end{lem}

% \begin{proof}[Proof Sketch] 
%   By induction on the structure of a correct proof of "\obl{\G, o \DEF
%     \hat e, \D ||- f}" or "\obl{\G, \hide{o \DEF \hat e}, \D ||- f}"
%   or "\obl{\G, \NEW o, \D ||- f}" whose leaf obligations are instances
%   of \tlatwo axioms..
% \end{proof}

\begin{lem}[Weakening] \label{thm:weaken}
  %
  If "\obl{\G, \D ||- e}" is true, then "\obl{\G, h, \D ||- e}" is true.
\end{lem}

% \begin{proof}[Proof Sketch]
%   By induction on the structure of a correct proof of "\obl{\G, \D ||-
%     e}" whose leaf obligations are instances of \tlatwo axioms.
% \end{proof}

As mentioned earlier, filtration makes the obligation implicationally
weaker.  That observation is a corollary 
%
% LL: I have no idea what the following parenthesized \ref is supposed
% to mean, so I omitted it.
% 
% (\ref{thm:filtration}) 
% 
of the following lemma.
%
% LL: I omitted the following from the previous clause because I
% couldn't understand it.  If someone wants to reinsert it, either
% reword it so it makes sense or else tell me what it's supposed to mean
% and I'll do it.
%
% with a strengthened induction invariant.

\begin{lem} \label{thm:filtration-lemma}
  %
  If "\obl{\filter{\G}, \D ||- e}" is true and "\obl{\D ||- e}" is
  filtered, then "\obl{\G, \D ||- e}" is true.
\end{lem}

\llnote{Will someone tell me what ``i.h.'' means.  (It's used several times in the proof.)
Does it need to be explained?}
\begin{proof}
  We proceed by induction on the structure of "\G".
  \begin{icom}
  \item [\s11.] \textit{Case} "\G" \textit{is} "\cdot" \Trivial
  \item [\s12.] \textit{Case} "\G" \textit{is} "\W, \NEW x"
    \begin{icom}
% LL: The following step was changed because of the removal of the formal def of 
%     filtered.
    \item [\s21.] "\obl{\NEW x, \D ||- e}" is filtered \by definition of \emph{filtered}.
% LL: The proof language uses QED, not Thus.  Since we numbered QED steps in the
%     example, we should do it here.
%     \item [\s2.] \textit{Thus}, "\G, \D ||- e" is true \by i.h. on "\W", \s21
    \item [\s22.] \Qed (which asserts "\obl{\G, \D ||- e}") \by i.h. on "\W", \s21
    \end{icom}
%   \item [\s13.] \textit{Cases of} "\G" being "\W, o \DEF \hat e" or "\W, \phi" are similar to \s12.
\end{icom}
The proofs of the next two statements are similar to that of \s12 and are omitted.
\begin{icom}
  \item [\s13.] \textit{Case} "\G" \textit{is} "\W, o \DEF \hat e" 
  \item [\s14.] \textit{Case} "\G" \textit{is} "\W, \phi" 
  \item [\s15.] \textit{Case} "\G" \textit{is} "\W, \hide{o \DEF \hat e}".
    \begin{icom}
    \item [\s21.] "\obl{\NEW o, \D ||- e}" is filtered \by definition of \textit{filtered}.
    \item [\s22.] "\obl{\W, \NEW o, \D ||- e}" is true \by i.h. on "\W", \s21
\llnote{Lemma~\ref{thm:op-subst} contains a hypothesis about bounding of free
identifiers.  Why is that hypothesis true?}
    \item [\s23.] \Qed 
% \textit{Thus}, "\G, \D ||- e" is true 
                 \by \s22, lemma~\ref{thm:op-subst}
    \end{icom}
  \item [\s16.] \textit{case} "\G" \textit{is} "\W, \hide{\phi}".
    \begin{icom}
    \item [\s21.] "\obl{\filter{\W}, \D ||- e}" is true \by definition of \textit{filtered}.
    \item [\s22.] "\obl{\W, \D ||- e}" is true \by i.h. on "\W", \s21
    \item [\s23.] \Qed
(which asserts "\obl{\W, \hide{\phi}, \D ||- e}") \by \s22, lemma~\ref{thm:weaken}
    \end{icom}
  \item [\s17.] \Qed 
%%%%    (all cases covered) 
                 \by \s11\,\dots\,\s16.
  \end{icom}
\end{proof}
An immediate corollary of Lemma~\ref{thm:filtration-lemma} is:

\begin{cor} \label{thm:filtration}
  % 
  If "\filter{\G ||- e}" is true, then "\obl{\G ||- e}" is true.
\end{cor}

% \begin{proof}
%   By lemma~\ref{thm:filtration-lemma} (for "\D = \cdot").
% \end{proof}

Obligations also admit a form of \textit{cut}, wherein a proof of a
sub-obligation can be used to discharge its uses as an assumption in a
different obligation.  The following lemma is proved by induction on the
structure of a correct proof.

\begin{lem}[Cut] \label{thm:cut}
  %
  If "\obl{\G, \D ||- e}" and "\obl{\G, (\D ||- e) ||- f}" are true,
  then "\obl{\G ||- f}" is true.
\end{lem}

% \begin{proof}[Proof Sketch]
%   By induction on the structure of a correct proof of "\obl{\G, (\D
%     ||- e) ||- f}" whose leaf obligations are instances of \tlatwo
%   axioms.
% \end{proof}

\llnote{I would recommend removing the following paragraph.  If not, it should
be made clear whether these lemmas \emph{have} been proved in Isabelle, or
whether we think they should be easy to prove in Isabelle.}

Formal proofs of these lemmas require defining the notion of a
\textit{primitive} \tlatwo proof that generates only obligations that
are instances of the axioms of \tlatwo and then reasoning by induction
on such proofs.  These lemmas are simple facts in minimal logic 
("=>,\forall") and are easily proved by Isabelle.

% We have not given the full details of the unsurprising
% lemmas~\ref{thm:op-subst}, \ref{thm:delete}, \ref{thm:weaken},
% and~\ref{thm:cut}; a more formal development will have to define the
% notion of a \textit{primitive} \tlatwo proof that only generates
% obligations that are instances of the axioms of \tlatwo, and then
% reason by induction on such proofs. Indeed, these lemmas are trivial
% facts in minimal logic ("=>, \forall") and easily discharged by
% Isabelle.

We finish our description of context management by describing how
making definitions usable and hiding them with \USE\ and \HIDE\ steps
is handled.  We denote such using and hiding with the terms
\kwd{using} and \kwd{hiding}.  Recall that the truth value of an
obligation is changed when hidden facts and definitions are removed by
filtering, not when they are made usable or hidden.

\llnote{Kaustuv, this definition needs to be completely rewritten.  
Here are some of its problems:\\
-- The term ``context'' is misused, since a context contains a current goal.
The correct term is probably \emph{assumption} or \emph{sequence of assumptions}.
You should check the terminology that's used in Section 3.  It may be useful
to add a term for it (and use it in Section 3). \\
-- The use of equality is bizarre.  If what you're calling a ``context'' is
a sequence of assumptions, then you have proof rules whose hypothesis
states that $\Delta$ is equal to one thing and whose conclusion states that $\Delta$
is equal to something else.  \\
-- The use of $h \in\ldots$ is less objectionable, but
is obviously some sort of informal notation.  It looks like you're
using pedantic notation to try to pretend that something completely informal
is actually a formal definition.\\
-- This is called a definition, but there is no indication of what it's a definition
of.  You seem to be defining an algorithm.  I suggest that you remove the ``Definition''
label and just describe the algorithm in text. \vspace{-1.6\baselineskip}
}

\begin{defn} \label{defn:use/hide}
  %
  If "\G" is a context and "\P" a list of operator names, then:
  \begin{ecom}
  \item "\G \USING \P" is a context "\D" given by induction on "\P" as follows:
    \begin{gather*}
      \I{"\G_0, h, \G_1 \USING \P, o = \D"}
        {"\G_0, h, \G_1 \USING \P = \D"
         &
         h \in \{ \NEW o, o \DEF \hat e \}
        }
      \LSP
      \I{"\G_0, \hide{o \DEF \hat e}, \G_1 \USING \P, o = \D"}
        {"\G_0, o \DEF \hat e, \G_1 \USING \P = \D"}
    \end{gather*}
  \item "\G \HIDING \P" is a context "\D" given by induction on "\P" as follows:
    \begin{gather*}
      \I{"\G_0, h, \G_1 \HIDING \P, o = \D"}
        {"\G_0, h, \G_1 \HIDING \P = \D"
         &
         h \in \{ \NEW o, \hide{o \DEF \hat e} \}
        }
      \LSP
      \I{"\G_0, o \DEF \hat e, \G_1 \HIDING \P, o = \D"}
        {"\G_0, \hide{o \DEF \hat e}, \G_1 \HIDING \P = \D"}
    \end{gather*}
  \end{ecom}
\end{defn}

\llnote{Why is the following a fact and not a theorem/lemma?  
Where is its proof? \vspace{-1.6\baselineskip}}

\begin{fac} \label{thm:use/hide}
  %
  "\obl{\G ||- e}" is true iff "\obl{\D ||- e}" is true where "\G
  \USING \P = \D" or "\G \HIDING \P = \D". \qed
\end{fac}

\section{Computing Obligations}
\label{sec:appendix.proof-transformation}

As mentioned in section~\ref{sec:obligations}, the \PM uses a \tlatwo
proof to perform a pair of computations on obligations:
\textit{checking} and \textit{transformation}. Each procedure is
declaritively specified below. For the purposes of this section, we
shall treat an unlabelled step start token "\s{n}" as a labelled start
token "\s{n}{l}" with "l" empty.

\llnote{We've already called "\pi : \obl{\G ||- e}" a claim.  We can't
now say that writing it means something new.  Also, I have no idea
what it's supposed to mean to derive this claim from proof rules.
We've already said a derivation step works from the conclusion to the
hypotheses.  How on earth are we supposed to derive a single claim
"\pi : \obl{\G ||- e}" using proof rules that have more than one
hypothesis?  }

\begin{defn}[Proof checking] \label{defn:checking}
  % 
  The \tlatwo proof "\pi" \emph{checks} an obligation "\obl{\G ||-
    e}", written "\pi : \obl{\G ||- e}", if there is a derivation of
  "\pi : \obl{\G ||- e}" using the three inference rules below such
  that the leaves of the derivation have one of the two primitive
  forms "\OBVIOUS : (\W ||- g)" and "\OMITTED : (\W ||- g)".
  \begin{gather*}
    \I[\BY]{"\BY\ \Phi\ \DEFS\ \P : (\G
      ||- e)"}
      {"\s{\infty}.\ \USE\ \Phi\ \DEFS\ \P" : "(\G ||- e) --> (\D ||- f)"
       &
       "\OBVIOUS : (\D ||- f)"}
    \SP
    \I[\QED]{"\s{n}l.\ \QED\ \PROOF\ \pi : (\G ||- e)"}
      {"\pi : (\G ||- e)"}
    \\
    \I[non-\QED]{"\s{n}l_1.\ \sigma_1\ \s{n}l_2.\ \sigma_2\ \dotso \s{n}l_m.\ \sigma_m : (\G ||- e)"}
      {"m > 1"
       &
       "\s n l_1.\ \sigma_1 : (\G ||- e) --> (\D ||- f)"
       &
       "\s{n}l_2.\ \sigma_2\ \dotso\ \s{n}l_m.\ \sigma_m : (\D ||- f)"}
  \end{gather*}  
\end{defn}

\llnote{We've already said that a proof obligation is a primitive claim.
You're now using ``obligation'' to mean something else.  You need to make the 
notation of the appendix consistent with that of the main body.  You also need
to add some prose indicating how this formalism relates to the informal
discussion in the main text.  In particular, nowhere is the hypothesis that
the claim is meaningful mentioned.\\[.3em]
Also, you are still using ``='' here in a way that strikes me as rather informal.
Here, it seems to be used as a syntactic precondition, not really as a hypothesis.
In particular, the algorithm for generating proof obligations involves working
from conclusion to hypotheses.  However, a ``hypothesis'' like the first one
in the first rule doesn't lead to any proof obligations---it's a precondition
for applying the rule.\vspace{-.65\baselineskip}}

\begin{defn}[Proof transformation] \label{defn:transform}
  %
  The \tlatwo proof step "\s{n}l.\ \sigma" \emph{transforms} the
  obligation "\obl{\G ||- e}" into "\obl{\D ||- f}", written "\s n l.\
  \sigma : \obl{\G ||- e} --> \obl{\D ||- f}".
  %
  \begin{gather*}
    \I[\USE\ \DEFS]{"\s n l.\ \USE\ \Phi\ \DEFS\ \P : \obl{\G ||- e} --> \obl{\D ||- f}"}
      {"\G \USING \P = \W"
       &
       "\s n l.\ \USE\ \Phi : \obl{\W ||- e} --> \obl{\D ||- f}"}
    \\
    \I[\HIDE\ \DEFS]{"\s n l.\ \HIDE\ \Phi\ \DEFS\ \P : \obl{\G ||- e} --> \obl{\D ||- f}"}
      {"\G \HIDING \P = \W"
       &
       "\s n l.\ \HIDE\ \Phi : \obl{\W ||- e} --> \obl{\D ||- f}"}
    \\
    \I[\DEFINE]{"\s n l.\ \DEFINE\ o \DEF \hat e : \obl{\G ||- e} --> \obl{\G, \hide{o \DEF \hat e} ||- e}"}
    \\
    \I["\USE_0"]{"\s n l.\ \USE\ \cdot : \obl{\G ||- e} --> \obl{\G ||- e}"}
    \SP
    \I["\HIDE_0"]{"\s n l.\ \HIDE\ \cdot : \obl{\G ||- e} --> \obl{\G ||- e}"}
    \\
    \I["\USE_1"]{"\s n l.\ \USE\ \obl{\G_0 ||- e_0}, \Phi : \obl{\G ||- e} --> \obl{\D ||- f}"}
      {"\OBVIOUS : \obl{\G, \G_0 ||- e_0}"
       &
       "\s n l.\ \USE\ \Phi : \obl{\G, \obl{\G_0 ||- e_0} ||- e} --> \obl{\D ||- f}"}
    \\
    \I["\HIDE_1"]{"\s n l.\ \HIDE\ \phi, \Phi : \obl{\G_0, \phi, \G_1 ||- e} --> \obl{\D ||- f}"}
      {"\s n l.\ \HIDE\ \Phi : \obl{\G_0, hide{\phi}, \G_1 ||- e} --> \obl{\D ||- f}"}
    \\
    \I["\TAKE_0"]{"\s n l.\ \TAKE\ \cdot : \obl{\G ||- e} --> \obl{\G ||- e}"}
    \LSP
    \I["\WITNESS_0"]{"\s n l.\ \WITNESS\ \cdot : \obl{\G ||- e} --> \obl{\G ||- e}"}
    \\
    \I["\TAKE_1"]{"\s n l.\ \TAKE\ u, \vec\beta : \obl{\G ||- \forall x : e} --> \obl{\D ||- f}"}
      {"\s n l.\ \TAKE\ \vec\beta : \obl{\G, \NEW u ||- e [x := u]} --> \obl{\D ||- f}"}
    \\
    \I["\TAKE_2"]{"\s n l.\ \TAKE\ u \in T, \vec\beta : \obl{\G ||- \forall x \in S : e} --> \obl{\D ||- f}"}
      {"\OBVIOUS : \obl{\G ||- S \subseteq T}"
       &
       "\s n l.\ \TAKE\ \vec\beta : \obl{\G, \NEW u, u \in T ||- e [x := u]} --> \obl{\D ||- f}"}
    \\
    \I["\WITNESS_1"]{"\s n l.\ \WITNESS\ w, \W : \obl{\G ||- \exists x : e} --> \obl{\D ||- f}"}
      {"\s n l.\ \WITNESS\ \W : \obl{\G ||- e [x := w]} --> \obl{\D ||- f}"}
    \\
    \I["\WITNESS_2"]{"\s n l.\ \WITNESS\ w \in T, \W : \obl{\G ||- \exists x \in S : e} --> \obl{\D ||- f}"}
      {\begin{array}[b]{c}
          "\OBVIOUS : \obl{\G ||- T \subseteq S}" \\
          "\OBVIOUS : \obl{\G ||- w \in T}"
       \end{array}
       &
       "\s n l.\ \WITNESS\ \W : \obl{\G, w \in T ||- e [x := w]} --> \obl{\D ||- f}"}
    \\
    \I[\HAVE]{"\s n l.\ \HAVE\ g : \obl{\G ||- e => f} --> \obl{\G, g ||- f}"}
      {"\OBVIOUS : \obl{\G, e ||- g}"}
    \\
    \I[\ASSERT]{"\s n l.\ \obl{\D ||- f}\ \PROOF\ \pi : \obl{\G_0 ||- e} --> {}"
        \left\{ \begin{array}{ll}
            "\obl{\G, \obl{\D ||- f} ||- e}" & \text{if "l" empty} \\
            "\obl{\G, \hide{\obl{\D ||- f}} ||- e}" & \text{otherwise}
        \end{array} \right.
      }
      {\G = \left\{\begin{array}{ll}
          \G_0 & \text{if "l" empty} \\
          \G_0, \s n l \DEF \obl{"\D ||- f"} & \text{otherwise}
       \end{array} \right.
       &
       \begin{array}{c} \\"\pi : \obl{\G, \hide{\lnot e}, \D ||- f}" \end{array}
      }
    \\
    \I[\CASE]{"\s n l.\ \CASE\ g\ \PROOF\ \pi : \obl{\G ||- e}
                     --> \obl{\D ||- f}"}
      {"\s n l.\ \obl{g ||- e}\ \PROOF\ \pi : \obl{\G ||- e} --> \obl{\D ||- f}"}
    \\
    \I[\SUFFICES]{"\s n l.\ \SUFFICES\ \obl{\D ||- f}\ \PROOF\ \pi : \obl{\G_0 ||- e} --> \obl{\G, \hide{\lnot e}, \D ||- f}"}
      {\G = \left\{\begin{array}{ll}
          \G_0 & \text{if "l" empty} \\
          \G_0, \s n l \DEF \obl{"\D ||- f"} & \text{otherwise}
       \end{array} \right.
       &
       \pi : 
       \left\{ \begin{array}{ll}
           "\obl{\G, \obl{\D ||- f} ||- e}" & \text{if "l" empty} \\
           "\obl{\G, \hide{\obl{\D ||- f}} ||- e}" & \text{otherwise}
       \end{array} \right.
      }
    \\
    \I[\PICK]{"\s n l.\ \PICK\ \vec\beta : p\ \PROOF\ \pi :
                \obl{\G ||- e} --> \obl{\G, \bigl\|\vec \beta\bigr\|, p ||- e}"}
      {"\pi : \obl{\G ||- \exists \vec \beta : p}"}
  \end{gather*}
\end{defn}

\ednote{KC}{Still need to define "\bigl\|\vec\beta\bigr\|" or find a way to avoid it.}

We now give a somewhat detailed proof of the completeness theorem.

\begin{thm}[Completeness] \mbox{}
  %
  \begin{ecom}
  \item If "\pi : \obl{\G ||- e}" is complete, then "\obl{\G ||- e}"
    is true.
  \item If "\sigma : \obl{\G ||- e} --> \obl{\D ||- f}" is complete and
    "\obl{\D ||- f}" is true, then "\obl{\G ||- e}" is true.
  \end{ecom}
\end{thm}

\begin{proof}
  By induction on the structure of the given complete derivations.
  %
  \begin{icom} \setlength{\itemsep}{6pt}
  \item[\s1.] \textit{Case }
    %
    $ \Ic[\BY.]{"\BY\ \Phi\ \DEFS\ \P : \obl{\G
      ||- e}"} {"\s{\infty}.\ \USE\ \Phi\ \DEFS\ \P : \obl{\G
      ||- e} --> \obl{\D ||- f}" & "\OBVIOUS : \obl{\D ||- f}" }$\\

    \begin{icom}
    \item[\s21.] "\obl{\D ||- f}" is true \by (1), completeness
    \item[\s2.] \textit{Thus} "\obl{\G ||- e}" is true \by (2), completeness,
      \s21
    \end{icom}

  \item[\s1.] \textit{Case }
    $\Ic[\QED]{"\s{n}l.\ \QED\ \PROOF\ \pi : (\G ||- e)"}{"\pi : (\G ||- e)"}$ \by (1), completeness

  \item[\s1.] \textit{Case }
    %
    $\Ic[non-\QED]{"\s{n}l_1.\ \sigma_1\ \s{n}l_2.\ \sigma_2\ \dotso \s{n}l_m.\ \sigma_m : (\G ||- e)"}
      {"\s n l_1.\ \sigma_1 : (\G ||- e) --> (\D ||- f)"
       &
       "\s{n}l_2.\ \sigma_2\ \dotso\ \s{n}l_m.\ \sigma_m : (\D ||- f)"}$ \\

    \begin{icom}
    \item[\s21.] "\obl{\D ||- f}" is true \by (1), completeness
    \item[\s2.] \textit{Thus} "\obl{\G ||- e}" is true \by (2), completeness,
      \s21
    \end{icom}
    
  \item[---] This covers all the cases for (1).

  \item[\s1.] \textit{Case }
    %
    $\Ic[\USE\ \DEFS.]{"\s n l.\ \USE\ \Phi\ \DEFS\ \P : \obl{\G ||- e} --> \obl{\D ||- f}"}
        {"\G \USING \P = \W"
         &
         "\s n l.\ \USE\ \Phi : \obl{\W ||- e} --> \obl{\D ||- f}"}$ \\

    \begin{icom}
    \item[\s21.] \textit{Assume} "\obl{\D ||- f}" is true
    \item[\s22.] "\obl{\W ||- e}" is true \by (2), assumption, \s21
    \item[\s2.] \textit{Thus} "\obl{\G ||- e}" is true \by \s22,
      fact~\ref{thm:use/hide}
    \end{icom}

  \item[\s1.] \textit{Case } "\HIDE\ \DEFS" is similar to $\<1\>$4.

  \item[\s1.] \textit{Case }
    %
    $\Ic[\DEFINE.]
        {"\s n l.\ \DEFINE\ o \DEF \hat e : \obl{\G ||- e}
            --> \obl{\G, \hide{o \DEF \hat e} ||- e}"}$ \\

    \begin{icom}
    \item[\s21.] \textit{Assume} "\obl{\G, \hide{o \DEF \hat e} ||- e}" is true
    \item[\s2.] \textit{Thus} "\obl{\G ||- e}" is true \by \s21, lemma~\ref{thm:delete}, closedness assumption
    \end{icom}
    
  \item[\s1.] \textit{Cases of} "\USE_0", "\HIDE_0", "\TAKE_0", and
    "\WITNESS_0" are trivial.

  \item[\s1.] \textit{Case }
    %
    $\Ic["\USE_1".]{"\s n l.\ \USE\ \obl{\G_0 ||- e_0}, \Phi : \obl{\G ||- e} --> \obl{\D ||- f}"}
        {"\OBVIOUS : \obl{\G, \G_0 ||- e_0}"
         &
         "\s n l.\ \USE\ \Phi : \obl{\G, \obl{\G_0 ||- e_0} ||- e} --> \obl{\D ||- f}"}$ \\

    \begin{icom}
    \item[\s21.] \textit{Assume} "\obl{\D ||- f}" is true
    \item[\s22.] "\obl{\G, \obl{\G_0 ||- e_0} ||- e}" is true \by (2),
      completeness, \s21
    \item[\s23.] "\obl{\G, \G_0 ||- e_0}" is true \by (1), completeness
    \item[\s2.] \textit{Thus} "\obl{\G ||- e}" is true \by \s22, \s23,
      cut (thm.~\ref{thm:cut})
    \end{icom}

  \item[\s1.] \textit{Case }
    %
    $\Ic["\HIDE_1".]{"\s n l.\ \HIDE\ \phi, \Phi : \obl{\G_0, \phi, \G_1 ||- e} --> \obl{\D ||- f}"}
        {"\s n l.\ \HIDE\ \Phi : \obl{\G_0, hide{\phi}, \G_1 ||- e} --> \obl{\D ||- f}"}$ \\

    \begin{icom}
    \item[\s21.] \textit{Assume} "\obl{\D ||- f}" is true
    \item[\s22.] "\obl{\G_0, \hide{\phi}, \G_1 ||- e}" is true \by (2),
      completeness, \s21
    \item[\s2.] \textit{Thus} "\obl{\G_0, \phi, \G_1 ||- e}" is true
      \by \s22
    \end{icom}

  \item[\s1.] \textit{Case }
    % 
    $\Ic["\TAKE_1".]{"\s n l.\ \TAKE\ u, \vec\beta : \obl{\G ||- \forall x : e} --> \obl{\D ||- f}"}
        {"\s n l.\ \TAKE\ \vec\beta : \obl{\G, \NEW u ||- e [x := u]} --> \obl{\D ||- f}"}$ \\

    \begin{icom}
    \item[\s21.] \textit{Assume} "\obl{\D ||- f}" is true
    \item[\s22.] "\obl{\G, \NEW u ||- e [x := u]}" is true \by (2), completeness,
      \s21
    \item[\s2.] \textit{Thus} "\obl{\G ||- \forall x : e}" is true \by \s22
    \end{icom}

  \item[\s1.] \textit{Case }
    %
    $\Ic["\TAKE_2".]{"\s n l.\ \TAKE\ u \in T, \vec\beta : \obl{\G ||- \forall x \in S : e} --> \obl{\D ||- f}"}
      {"\OBVIOUS : \obl{\G ||- S \subseteq T}"
         &
         "\s n l.\ \TAKE\ \vec\beta : \obl{\G, \NEW u, u \in T ||- e [x := u]} --> \obl{\D ||- f}"}$ \\

    \begin{icom}
    \item[\s21.] \textit{Assume} "\obl{\D ||- f}" is true
    \item[\s22.] "\obl{\G, \NEW u, u \in S ||- e[x := u]}" is true
      \begin{icom}
      \item[\s31.] "\obl{\G, \NEW u, u \in T ||- e [x := u]}" is true
        \by (2), completeness, \s21
      \item[\s32.] "\obl{\G, \NEW u, u \in S ||- u \in T}" \by (1),
        completeness, \textit{defn.} of $\subseteq$
      \item[\s3.] \Qed \by \s31, \s32, cut
      \end{icom}
    \item[\s2.] \textit{Thus} "\obl{\G ||- \forall x \in S : e}" is true \by \s22
    \end{icom}

  \item[\s1.] \textit{Case }
    %
    $\Ic["\WITNESS_1".]{"\s n l.\ \WITNESS\ w, \W : \obl{\G ||- \exists x : e} --> \obl{\D ||- f}"}
        {"\s n l.\ \WITNESS\ \W : \obl{\G ||- e [x := w]} --> \obl{\D ||- f}"}$ \\

    \begin{icom}
    \item[\s21.] \textit{Assume} "\obl{\D ||- f}" is true
    \item[\s22.] "\obl{\G ||- e [x := w]}" is true \by (2), completeness,
      \s21
    \item[\s2.] \textit{Thus} "\obl{\G ||- \exists x : e}" is true \by \s22
    \end{icom}

  \item[\s1.] \textit{Case }
    %
    $\Im["\WITNESS_2".]{"\s n l.\ \WITNESS\ w \in T, \W : \obl{\G ||- \exists x \in S : e} --> \obl{\D ||- f}"}
        {\begin{array}[b]{c}
            "\OBVIOUS : \obl{\G ||- T \subseteq S}" \\
            "\OBVIOUS : \obl{\G ||- w \in T}"
         \end{array}
         &
         "\s n l.\ \WITNESS\ \W : \obl{\G, w \in T ||- e [x := w]} --> \obl{\D ||- f}"}$ \\

    \begin{icom}
    \item[\s21.] \textit{Assume} "\obl{\D ||- f}" is true
    \item[\s22.] "\obl{\G ||- w \in S}" is true
      \begin{icom}
      \item[\s31.] "\obl{\G, w \in T ||- w \in S}" is true \by (1),
        \textit{defn}. of $\subseteq$
      \item[\s3.] \Qed \by \s31, (1), cut
      \end{icom}
    \item[\s{2}{3}.] "\obl{\G ||- e [x := w]}" is true
      \begin{icom}
      \item[\s{3}{1}.] "\obl{\G, w \in T ||- e [x := w]}" is true \by
        (2), completeness, \s21
      \item[\s{3}.] \Qed \by \s31, (1), cut
      \end{icom}
    \item[\s{2}.] \textit{Thus} "\obl{G ||- \exists x \in S : e}" is
      true \by \s22, \s23
    \end{icom}

  \item[\s1.] \textit{Case }
    %
    $\Ic[\HAVE.]{"\s n l.\ \HAVE\ g : \obl{\G ||- e => f} --> \obl{\G, g ||- f}"}
        {"\OBVIOUS : \obl{\G, e ||- g}"}$ \\

    \begin{icom}
    \item[\s21.] \textit{Assume} "\obl{\G, g ||- f}" is true
    \item[\s22.] "\obl{\G, e, g ||- f}" is true \by \s21, weakening (thm.~\ref{thm:weaken})
    \item[\s23.] "\obl{\G, e ||- g}" is true \by (1)
    \item[\s24.] "\obl{\G, e ||- f}" is true \by \s22, \s23,
      cut
    \item[\s2.] \textit{Thus} "\G ||- e => f" is true \by \s24
    \end{icom}

  \item[\s1.] \textit{Case }
    %
    $\Ic[\ASSERT.]{"\s n l.\ \obl{\D ||- f}\ \PROOF\ \pi : \obl{\G_0 ||- e} --> {}"
          \left\{ \begin{array}{ll}
              "\obl{\G, \obl{\D ||- f} ||- e}" & \text{if "l" empty} \\
              "\obl{\G, \hide{\obl{\D ||- f}} ||- e}" & \text{otherwise}
          \end{array} \right.
        }
        {\G = \left\{\begin{array}{ll}
            \G_0 & \text{if "l" empty} \\
            \G_0, \s n l \DEF \obl{"\D ||- f"} & \text{otherwise}
         \end{array} \right.
         &
         \begin{array}{c} \\"\pi : \obl{\G, \hide{\lnot e}, \D ||- f}" \end{array}
        }$ \\

    \begin{icom}
    \item[\s2.] \textit{Case} "l" empty
      \begin{icom}
      \item[\s31.] \textit{Assume} "\obl{\G, \obl{\D ||- f} ||- e}" is true
      \item[\s32.] "\obl{\G, \hide{\lnot e}, \obl{\D ||- f} ||- e}" \by \s31, weakening
      \item[\s33.] "\obl{\G, \hide{\lnot e}, \D ||- f}" is true \by (1), completeness
      \item[\s34.] "\obl{\G, \hide{\lnot e} ||- e}" is true \by \s32, \s33, cut
      \item[\s3.] \textit{Thus} "\G ||- e" is true \by \s34, Peirce's law
      \end{icom}
    \item[\s2.] \textit{Case} "l" non-empty
      \begin{icom}
      \item[\s31.] \textit{Assume} "\obl{\G, \hide{\obl{\D ||- f}} ||- e}" is true
      \item[\s32.] "\obl{\G, \hide{\lnot e}, \hide{\obl{\D ||- f}} ||- e}" \by \s31, weakening
      \item[\s33.] "\obl{\G, \hide{\lnot e}, \D ||- f}" is true \by (1), completeness
      \item[\s34.] "\obl{\G, \hide{\lnot e} ||- e}" is true \by \s32, \s33, cut
      \item[\s3.] \textit{Thus} "\G ||- e" is true \by \s34, Peirce's law
      \end{icom}
    \item[\s2.] \Qed (all cases covered)
    \end{icom}

    \ednote{KC}{Peirce's law not defined, and the instance "(\lnot e
      => e) => e" of Peirce's law goes by many other names.}

  \item[\s1.] \textit{Cases of } \CASE and \SUFFICES are similar to that of \ASSERT.

  \item[\s1.] \textit{Case }
    %
    $\Ic[\PICK.]{"\s n l.\ \PICK\ \vec\beta : p\ \PROOF\ \pi :
                \obl{\G ||- e} --> \obl{\G, \bigl\|\vec \beta\bigr\|, p ||- e}"}
      {"\pi : \obl{\G ||- \exists \vec \beta : p}"}$ \\

    \begin{icom}
    \item[\s21.] \textit{Assume} "\obl{\G, \bigl\| \vec \beta \bigr\|, p ||- e}" is true
    \item[\s22.] "\obl{\G, \exists \vec \beta : p ||- e}" \by \s21,
      $\exists_L$
      \ednote{KC}{Need to define $\exists_L$?}
    \item[\s23.] "\obl{\G ||- \exists \vec \beta : p}" is true \by (1), completeness
    \item[\s2.] \textit{Thus} "\obl{\G ||- e}" is true \by \s22, \s33,
      cut
    \end{icom}

  \item[\s1.] \Qed (all inductive cases covered) \qedhere
  \end{icom}
\end{proof}

%%%% PLEASE DO NOT EDIT BELOW THIS LINE
\ifx\master\undefined
{\let\master\relax%%% -*- mode: LaTeX; TeX-master: "main.tex"; -*-

\ifx\master\undefined
\documentclass[a4paper]{easychair}
\usepackage{submission}
\begin{document}
{\let\master\relax %%% -*- mode: LaTeX; TeX-master: "main.tex"; -*-

\ifx\master\undefined
\documentclass[a4paper]{easychair}
\usepackage{submission}
\begin{document}
\fi
%%%% PLEASE DO NOT EDIT ABOVE THIS LINE

\title{A \tlaplus Proof System}

\titlerunning{A \tlaplus Proof System}

% \volumeinfo
% 	{P. Rudnicki, G. Sutcliffe} % editors
% 	{2}                         % number of editors
% 	{KEAPPA 2008}               % event
% 	{1}                         % volume
% 	{1}                         % issue
% 	{1}                         % starting page number


%  Alphabetically by surname
\author{
  Kaustuv Chaudhuri \\
  INRIA \\
  \and
  Damien Doligez \\
  INRIA \\
  \and
  Leslie Lamport \\
  Microsoft Research \\
  \and
  Stephan Merz \\
  INRIA \& Loria
}

\authorrunning{Chaudhuri, Doligez, Lamport, and Merz}

\maketitle

%%%% PLEASE DO NOT EDIT BELOW THIS LINE
\ifx\master\undefined
{\let\master\relax %%% -*- mode: LaTeX; TeX-master: "main.tex"; -*-

\ifx\master\undefined
\documentclass[a4paper]{easychair}
\usepackage{submission}
\begin{document}
{\let\master\relax \input{frontmatter}}
\fi
%%%% PLEASE DO NOT EDIT ABOVE THIS LINE

\bibliographystyle{plain}
\bibliography{submission}

%%%% PLEASE DO NOT EDIT BELOW THIS LINE
\ifx\master\undefined
\end{document}
\fi

% LocalWords:  tex Paxos
}
\end{document}
\fi

% LocalWords:  tex Rudnicki Sutcliffe KEAPPA Kaustuv Chaudhuri INRIA Doligez
% LocalWords:  Merz Loria
}
\fi
%%%% PLEASE DO NOT EDIT ABOVE THIS LINE

\bibliographystyle{plain}
\bibliography{submission}

%%%% PLEASE DO NOT EDIT BELOW THIS LINE
\ifx\master\undefined
\end{document}
\fi

% LocalWords:  tex Paxos
}
\end{document}
\fi
