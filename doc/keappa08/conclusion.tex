%%% -*- mode: LaTeX; TeX-master: "main.tex"; -*-

\ifx\master\undefined
\documentclass[a4paper]{easychair}
\usepackage{submission}
\begin{document}
{\let\master\relax%%% -*- mode: LaTeX; TeX-master: "main.tex"; -*-

\ifx\master\undefined
\documentclass[a4paper]{easychair}
\usepackage{submission}
\begin{document}
\fi
%%%% PLEASE DO NOT EDIT ABOVE THIS LINE

\title{A \tlaplus Proof System}

\titlerunning{A \tlaplus Proof System}

% \volumeinfo
% 	{P. Rudnicki, G. Sutcliffe} % editors
% 	{2}                         % number of editors
% 	{KEAPPA 2008}               % event
% 	{1}                         % volume
% 	{1}                         % issue
% 	{1}                         % starting page number


%  Alphabetically by surname
\author{
  Kaustuv Chaudhuri \\
  INRIA \\
  \and
  Damien Doligez \\
  INRIA \\
  \and
  Leslie Lamport \\
  Microsoft Research \\
  \and
  Stephan Merz \\
  INRIA \& Loria
}

\authorrunning{Chaudhuri, Doligez, Lamport, and Merz}

\maketitle

%%%% PLEASE DO NOT EDIT BELOW THIS LINE
\ifx\master\undefined
{\let\master\relax %%% -*- mode: LaTeX; TeX-master: "main.tex"; -*-

\ifx\master\undefined
\documentclass[a4paper]{easychair}
\usepackage{submission}
\begin{document}
{\let\master\relax %%% -*- mode: LaTeX; TeX-master: "main.tex"; -*-

\ifx\master\undefined
\documentclass[a4paper]{easychair}
\usepackage{submission}
\begin{document}
\fi
%%%% PLEASE DO NOT EDIT ABOVE THIS LINE

\title{A \tlaplus Proof System}

\titlerunning{A \tlaplus Proof System}

% \volumeinfo
% 	{P. Rudnicki, G. Sutcliffe} % editors
% 	{2}                         % number of editors
% 	{KEAPPA 2008}               % event
% 	{1}                         % volume
% 	{1}                         % issue
% 	{1}                         % starting page number


%  Alphabetically by surname
\author{
  Kaustuv Chaudhuri \\
  INRIA \\
  \and
  Damien Doligez \\
  INRIA \\
  \and
  Leslie Lamport \\
  Microsoft Research \\
  \and
  Stephan Merz \\
  INRIA \& Loria
}

\authorrunning{Chaudhuri, Doligez, Lamport, and Merz}

\maketitle

%%%% PLEASE DO NOT EDIT BELOW THIS LINE
\ifx\master\undefined
{\let\master\relax \input{rearmatter}}
\end{document}
\fi

% LocalWords:  tex Rudnicki Sutcliffe KEAPPA Kaustuv Chaudhuri INRIA Doligez
% LocalWords:  Merz Loria
}
\fi
%%%% PLEASE DO NOT EDIT ABOVE THIS LINE

\bibliographystyle{plain}
\bibliography{submission}

%%%% PLEASE DO NOT EDIT BELOW THIS LINE
\ifx\master\undefined
\end{document}
\fi

% LocalWords:  tex Paxos
}
\end{document}
\fi

% LocalWords:  tex Rudnicki Sutcliffe KEAPPA Kaustuv Chaudhuri INRIA Doligez
% LocalWords:  Merz Loria
}
\fi
%%%% PLEASE DO NOT EDIT ABOVE THIS LINE

\section{Conclusions and Future Work}
\label{sec:conclusions}

We have presented a hierarchically structured proof language for
\tlaplus.  It has several important features that help in managing the
complexity of proofs.  The hierarchical structure means that changes
made at any level of a proof are contained inside that level, which
helps construct and maintain proofs.  Leaf proofs can be omitted and
the resulting incomplete proof can be checked.  This allows different parts
of the proof to be written separately, in a non-linear fashion.  
%%
%%  LL: I deleted 
% 
%    All other proof systems that we know of enforce a linear proof
%    style in which steps that have not yet been proved can be used
%    only if explicitly added as hypotheses.
% 
% Because the \emph{cheat} operator of Isar makes it false.
% 
The more traditional linear proof style, in which steps that have not
yet been proved can be used only if explicitly added as hypotheses,
encourages proofs that use many separate lemmas.  Such proofs lack the
coherent structure of a single hierarchical proof.

The proof language lets the user freely and repeatedly make facts and
definitions usable or hidden.  Explicitly stating what is being used
to prove each step makes the proof easier for a human to understand.
It also aids a back-end prover by limiting its search for a proof
to ones that use only necessary facts.

% \llnote{I added a citation to the Mizar web page.  Would someone please
% try to find out if there's a better reference?\\[.4em]
% SM: Reference added (it's pretty old, but I couldn't find a better one).}

There are other declarative proof languages that to some extent are
similar to \tlatwo.  Mizar~\cite{rudnicki:mizar} and Isar~\cite{isar} encourage
a traditional natural deduction style in which what has been proved
comes at the end rather than being stated as the goal of a lower-level
proof.  The Focal Proof Language~\cite{focal} is essentially a subset
of the \tlatwo\ proof language.  Our experience with hierarchical
proofs in Focal provides additional confidence in the attractiveness
of our approach.  We know of no declarative proof language that has as
flexible a method of using and hiding facts and definitions as
\tlatwo.

The \PM\ transforms a proof into a collection of proof obligations to
be verified by a back-end prover.  Its current version handles proofs
of theorems in the non-temporal fragment of \tlaplus\ that do not
involve module instantiation (importing of modules with substitution).
Even with this limitation, the system can be useful for many
engineering applications.  We are therefore concentrating on making
the \PM\ and its back-end provers handle this fragment of \tlaplus\
effectively before extending them to the complete language.  The major
work that remains to be done on this is to complete the Zenon and
Isabelle inference rules for reasoning about the built-in constant
operators of \tlaplus.  There are also a few non-temporal aspects of
the \tlatwo\ language that the \PM\ does not yet handle, such as
subexpression naming.  We also expect to extend the \PM\ to support
additional back-end provers, including decision procedures for
arithmetic and for propositional temporal logic.

%% I commented that paragraph because it is almost entirely redundant
%% with what has been said at several points in the paper and is not
%% really "concluding" anything. A reader who hasn't read the rest of
%% the paper will get little out ot it. Plus it lets the neighbouring
%% paragraphs flow into each other much better. -- KC
%%
%% LL: I sympathize with the sentiment, often omitting concluding 
%%     sections myself.  However, as was pointed out by someone (you?),
%%     one purpose of a conclusion is to summarize what we've said, which
%%     we need to do because at many (most?) reviewers will not read more 
%%     than the intro and the conclusion.  The first sentence repeats what
%%     has been said.  The second states two facts that appeared in
%%     different places.  I believe every other sentence says something
%%     not stated earlier.


% \llnote{Is the last sentence accurate?  In particular, if the decision
%   procedure will be called through Isabelle, does it constitute an
%   additional prover?  Do we need to clarify this?}
% 
% \ednote{DD}{I think it does count as an additional prover. Whether it
%   is called through Isabelle is an implementation detail, so we can
%   leave this sentence as it is.}
% 
% \ednote{SM}{I agree.}

We are satisfied with the proof language and do not anticipate that
any major changes will be needed.  We do expect some minor tuning as
we get more experience using it.  (For example, we are not sure
whether local definitions should be usable by default.)  A graphical
user interface is being planned for the \tlaplus\ tools, including the
\PM. It will support the non-linear development of proofs that the
language and the proof system allow.

%%%% PLEASE DO NOT EDIT BELOW THIS LINE
\ifx\master\undefined
{\let\master\relax%%% -*- mode: LaTeX; TeX-master: "main.tex"; -*-

\ifx\master\undefined
\documentclass[a4paper]{easychair}
\usepackage{submission}
\begin{document}
{\let\master\relax %%% -*- mode: LaTeX; TeX-master: "main.tex"; -*-

\ifx\master\undefined
\documentclass[a4paper]{easychair}
\usepackage{submission}
\begin{document}
\fi
%%%% PLEASE DO NOT EDIT ABOVE THIS LINE

\title{A \tlaplus Proof System}

\titlerunning{A \tlaplus Proof System}

% \volumeinfo
% 	{P. Rudnicki, G. Sutcliffe} % editors
% 	{2}                         % number of editors
% 	{KEAPPA 2008}               % event
% 	{1}                         % volume
% 	{1}                         % issue
% 	{1}                         % starting page number


%  Alphabetically by surname
\author{
  Kaustuv Chaudhuri \\
  INRIA \\
  \and
  Damien Doligez \\
  INRIA \\
  \and
  Leslie Lamport \\
  Microsoft Research \\
  \and
  Stephan Merz \\
  INRIA \& Loria
}

\authorrunning{Chaudhuri, Doligez, Lamport, and Merz}

\maketitle

%%%% PLEASE DO NOT EDIT BELOW THIS LINE
\ifx\master\undefined
{\let\master\relax %%% -*- mode: LaTeX; TeX-master: "main.tex"; -*-

\ifx\master\undefined
\documentclass[a4paper]{easychair}
\usepackage{submission}
\begin{document}
{\let\master\relax \input{frontmatter}}
\fi
%%%% PLEASE DO NOT EDIT ABOVE THIS LINE

\bibliographystyle{plain}
\bibliography{submission}

%%%% PLEASE DO NOT EDIT BELOW THIS LINE
\ifx\master\undefined
\end{document}
\fi

% LocalWords:  tex Paxos
}
\end{document}
\fi

% LocalWords:  tex Rudnicki Sutcliffe KEAPPA Kaustuv Chaudhuri INRIA Doligez
% LocalWords:  Merz Loria
}
\fi
%%%% PLEASE DO NOT EDIT ABOVE THIS LINE

\bibliographystyle{plain}
\bibliography{submission}

%%%% PLEASE DO NOT EDIT BELOW THIS LINE
\ifx\master\undefined
\end{document}
\fi

% LocalWords:  tex Paxos
}
\end{document}
\fi
